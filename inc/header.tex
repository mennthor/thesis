%------------------------------------------------------------------------------
%-- Almost all settings and tweaks taken from: github.com/maxnoe/tudothesis
%------------------------------------------------------------------------------

\documentclass[
  headsepline,
  bibliography=totoc,
  parskip=half,
  cleardoublepage=plain,
  captions=tableheading,
  titlepage=firstiscover,
  headings=normal
  ]{scrbook}

% Warning, if another latex run is needed
\usepackage[aux]{rerunfilecheck}

% Just list chapters and sections in the toc, not subsections or smaller
\setcounter{tocdepth}{1}

% KOMA Headers
\usepackage{scrlayer-scrpage}
\pagestyle{scrheadings}

%------------------------------------------------------------------------------
%-- Language and Type
%------------------------------------------------------------------------------
\usepackage{fontspec}
\defaultfontfeatures{Ligatures=TeX}  % -- becomes en-dash etc.

% Language: english | german
\usepackage{polyglossia}
\setdefaultlanguage{english}
% For secondary abstract in german
\setotherlanguages{german}

% Intelligent quotation marks, language and nesting sensitive
\usepackage[autostyle]{csquotes}

% Microtypographical features, makes the text look nicer on the small scale
\usepackage{microtype}

%------------------------------------------------------------------------------
%-- Math
%------------------------------------------------------------------------------
\usepackage{amsmath}
\usepackage{amssymb}
\usepackage{mathtools}

% Enable Unicode-Math and follow the ISO-Standards for typesetting math
\usepackage[
  math-style=ISO,
  bold-style=ISO,
  sans-style=italic,
  nabla=upright,
  partial=upright,
]{unicode-math}
\setmathfont{Latin Modern Math}

% Small fracs for the text with \sfrac{}{}
\usepackage{xfrac}
\usepackage{braket}

%------------------------------------------------------------------------------
%-- Numbers and Units
%------------------------------------------------------------------------------
\usepackage[
  locale=US,
  separate-uncertainty=true,
  per-mode=symbol-or-fraction,
  exponent-product=\cdot,
]{siunitx}
\sisetup{math-micro=\text{µ},text-micro=µ}

%------------------------------------------------------------------------------
%--Tables
%------------------------------------------------------------------------------
\usepackage{booktabs}  % Use \toprule, \midrule, \bottomrule

%------------------------------------------------------------------------------
%-- Graphics
%------------------------------------------------------------------------------
\usepackage{graphicx}
\usepackage{grffile}  % Flexible filename support
\usepackage{pgf}

%------------------------------------------------------------------------------
%-- Floats
%------------------------------------------------------------------------------
% Allow figures to be placed in the running text by default:
\usepackage{scrhack}
\usepackage{float}
\floatplacement{figure}{htbp}
\floatplacement{table}{htbp}

% Keep figures and tables in the section
\usepackage[section, below]{placeins}

\usepackage{caption}
\captionsetup{
  labelfont=bf,
  format=plain,
  width=0.9\textwidth,
  font=small
  }
\usepackage{subcaption}

%------------------------------------------------------------------------------
%-- Customize list environments
%------------------------------------------------------------------------------
\usepackage{enumitem}

%------------------------------------------------------------------------------
%-- Colors
%------------------------------------------------------------------------------
\usepackage{xcolor}

%------------------------------------------------------------------------------
%-- Bibliography
%------------------------------------------------------------------------------
\usepackage[backend=biber]{biblatex}

%------------------------------------------------------------------------------
%-- Code environment
%------------------------------------------------------------------------------
\usepackage{listings}
  \lstset{basicstyle=\ttfamily}

%------------------------------------------------------------------------------
%-- Misc
%------------------------------------------------------------------------------
\usepackage[pdfusetitle,unicode]{hyperref}
\usepackage{bookmark}
\usepackage[shortcuts]{extdash}
\usepackage{scrlayer-scrpage}  % KOMA layout modifications

%------------------------------------------------------------------------------
%-- Use realtiv paths for standalone compiling subchapters
%------------------------------------------------------------------------------
\usepackage{import}

%------------------------------------------------------------------------------
%-- Custom commands
%------------------------------------------------------------------------------
\renewcommand{\d}[1]{\mathrm{d}#1}
\newcommand{\dd}[2]{\frac{\d{#1}}{\d{#2}}}
\newcommand{\deldel}[2]{\frac{\partial #1}{\partial #2}}

\newcommand{\mperiod}{\quad\text{.}}
\newcommand{\mcomma}{\quad\text{,}}
\newcommand{\mintertext}[1]{\quad\text{#1}\quad}

\DeclareMathOperator{\rect}{rect}

\newcommand{\CITE}[1]{\colorbox{green}{CITE: #1}}
\newcommand{\TODO}[1]{\colorbox{orange}{TODO: #1}}
\newcommand{\FIX}[1]{\colorbox{red}{\textcolor{white}{FIX: #1}}}

%------------------------------------------------------------------------------
%-- Custom title page options
%------------------------------------------------------------------------------
% Store the title in \thetitle, so one can access it in the document
\let\oldtitle\title%
\renewcommand{\title}[1]{\oldtitle{#1}\newcommand{\thetitle}{#1}}

% Initialize commands:
\newcommand\thebirthplace{}
\newcommand\thechair{}
\newcommand\thedivision{}
\newcommand\thethesisclass{}
\newcommand\thesubmissiondate{}
\newcommand\thefirstcorrector{}
\newcommand\thesecondcorrector{}

% New commands for information about the thesis
\newcommand\birthplace[1]{\renewcommand\thebirthplace{geboren in #1}}
\newcommand\chair[1]{\renewcommand\thechair{#1}}
\newcommand\division[1]{\renewcommand\thedivision{#1}}
\newcommand\thesisclass[1]{\renewcommand\thethesisclass{#1}}
\newcommand\submissiondate[1]{\renewcommand\thesubmissiondate{#1}}
\newcommand\firstcorrector[1]{\renewcommand\thefirstcorrector{#1}}
\newcommand\secondcorrector[1]{\renewcommand\thesecondcorrector{#1}}

% Pre-set title page elements
\subject{Arbeit zur Erlangung des akademischen Grades\\ \thethesisclass}
\publishers{\thechair \\ \thedivision \\ Technische Universität Dortmund}

% Page for the correctors
\newcommand{\makecorrectorpage}{%
  \thispagestyle{empty}
  \vspace*{\fill}
  \begin{tabular}{@{}l@{\hspace{2em}}l@{}}
      Erstgutachter:  & \thefirstcorrector \\
      Zweitgutachter: & \thesecondcorrector \\
      Abgabedatum:    & \thesubmissiondate
  \end{tabular}
  \newpage
}

% Add birthplace to author
\let\oldauthor\author
\renewcommand\author[1]{%
  \oldauthor{#1 \\ \thebirthplace}
}