\documentclass[parskip=half, cleardoublepage=plain]{scrbook}

% Warning, if another latex run is needed
\usepackage[aux]{rerunfilecheck}

% Just list chapters and sections in the toc, not subsections or smaller
\setcounter{tocdepth}{1}

%------------------------------------------------------------------------------
%-- Language and Type
%------------------------------------------------------------------------------
\usepackage{fontspec}
\defaultfontfeatures{Ligatures=TeX}  % -- becomes en-dash etc.

% Language: english | german
\usepackage{polyglossia}
\setdefaultlanguage{english}

% For german | english abstract and german | english titles in the toc
\setotherlanguages{english}

% Intelligent quotation marks, language and nesting sensitive
\usepackage[autostyle]{csquotes}

% Microtypographical features, makes the text look nicer on the small scale
\usepackage{microtype}

%------------------------------------------------------------------------------
%-- Math
%------------------------------------------------------------------------------
\usepackage{amsmath}
\usepackage{amssymb}
\usepackage{mathtools}

% Enable Unicode-Math and follow the ISO-Standards for typesetting math
\usepackage[
  math-style=ISO,
  bold-style=ISO,
  sans-style=italic,
  nabla=upright,
  partial=upright,
]{unicode-math}
\setmathfont{Latin Modern Math}

% Small fracs for the text with \sfrac{}{}
\usepackage{xfrac}
\usepackage{braket}

%------------------------------------------------------------------------------
%-- Numbers and Units
%------------------------------------------------------------------------------
\usepackage[
  locale=US,
  separate-uncertainty=true,
  per-mode=symbol-or-fraction,
  exponent-product=\cdot,
]{siunitx}
\sisetup{math-micro=\text{µ},text-micro=µ}

%------------------------------------------------------------------------------
%--Tables
%------------------------------------------------------------------------------
\usepackage{booktabs}  % Use \toprule, \midrule, \bottomrule

%------------------------------------------------------------------------------
%-- Graphics
%------------------------------------------------------------------------------
\usepackage{graphicx}
\usepackage{grffile}  % Flexible filename support

%------------------------------------------------------------------------------
%-- Floats
%------------------------------------------------------------------------------
% Allow figures to be placed in the running text by default:
\usepackage{scrhack}
\usepackage{float}
\floatplacement{figure}{htbp}
\floatplacement{table}{htbp}

% Keep figures and tables in the section
\usepackage[section, below]{placeins}

\usepackage{caption}
  \captionsetup{width=0.9\textwidth}
\usepackage{subcaption}

%------------------------------------------------------------------------------
%-- Customize list environments
%------------------------------------------------------------------------------
\usepackage{enumitem}

%------------------------------------------------------------------------------
%-- Colors
%------------------------------------------------------------------------------
\usepackage{xcolor}

%------------------------------------------------------------------------------
%-- Bibliography
%------------------------------------------------------------------------------
\usepackage[backend=biber]{biblatex}

%------------------------------------------------------------------------------
%-- Code environment
%------------------------------------------------------------------------------
\usepackage{listings}

%------------------------------------------------------------------------------
%-- Misc
%------------------------------------------------------------------------------
\usepackage[pdfusetitle,unicode]{hyperref}
\usepackage{bookmark}
\usepackage[shortcuts]{extdash}
\usepackage{scrlayer-scrpage}  % KOMA layout modifications

%------------------------------------------------------------------------------
%-- Todo notes (put .sty in $($kpsewhich -var-value TEXMFHOME)/tex/latex)
%------------------------------------------------------------------------------
\usepackage{mytodonotes}

%------------------------------------------------------------------------------
%-- Use realtiv paths for standalone compiling subchapters
%------------------------------------------------------------------------------
\usepackage{import}

%------------------------------------------------------------------------------
%-- Custom commands
%------------------------------------------------------------------------------
\renewcommand{\d}[1]{\mathrm{d}#1}
\newcommand{\dd}[2]{\frac{\d{#1}}{\d{#2}}}
\newcommand{\deldel}[2]{\frac{\partial #1}{\partial #2}}

\newcommand{\mperiod}{\quad\text{.}}
\newcommand{\mcomma}{\quad\text{,}}
\newcommand{\mintertext}[1]{\quad\text{#1}\quad}

\DeclareMathOperator{\rect}{rect}
