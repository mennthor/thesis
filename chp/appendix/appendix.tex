\chapter{Further references}
The appendix is mainly used to show additional plots, for example for distributions for other time windows than shown in the main matter.
These are often similar to each other and would otherwise clutter the main part too much.

The code written to enable the computational part of this analysis was mainly done in the \emph{Python} programming language, making heavy use of the scientific computing and visualization libraries \emph{numpy} \cite{numpy}, \emph{scipy} \cite{scipy}, \emph{matplotlib} \cite{matplotlib} and \emph{healpy} \cite{Gorski:2004by}.

\section{Notes on reproducibility}
\FIX{Shortly describe the folder and script structure?}

\section{Acknowledgements}
\FIX{Missing personal acknowledgements}


\chapter{Supplementary material}
\section{Datasets}
This appendix chapter includes additional plots for chapter~\ref{chp:datasets}.

% %%%%%%%%%%%%%%%%%%%%%%%%%%%%%%%%%%%%%%%%%%%%%%%%%%%%%%%%%%%%%%%%%%%%%%%%%%%%%
% HESE LLH maps skymap
% %%%%%%%%%%%%%%%%%%%%%%%%%%%%%%%%%%%%%%%%%%%%%%%%%%%%%%%%%%%%%%%%%%%%%%%%%%%%%
\begin{figure}[htbp]
  \centering
  \includegraphics{../datasets/plots/hese_events_reco_landscape_skymap.pdf}
  \caption[Combined Likelihood skymap of the 22 HESEs]{
    Skymap in equatorial coordinates and Mollweide projection of the combination of all Likelihood reconstruction maps from the track-like high energy starting events.
  }
  \label{fig:hese_events_reco_landscape_skymap}
\end{figure}


% %%%%%%%%%%%%%%%%%%%%%%%%%%%%%%%%%%%%%%%%%%%%%%%%%%%%%%%%%%%%%%%%%%%%%%%%%%%%%
% Eff areas and sindec distributions
% %%%%%%%%%%%%%%%%%%%%%%%%%%%%%%%%%%%%%%%%%%%%%%%%%%%%%%%%%%%%%%%%%%%%%%%%%%%%%
\begin{figure}[htbp]
  \centering
  \includegraphics{../datasets/plots/effA_and_sindec.pdf}
  \caption[Effective areas and $\sin(\delta_\nu)$ distributions]{
    Effective areas and $\sin(\delta_\nu)$ distributions for for the samples IC79, IC86, 2011, IC86, 2012--2014 and IC86, 2015 in the muon neutrino test dataset.
    The effective areas are shown for multiple declination bands to capture the main features of the underlying sample selection efficiencies.
    The $\sin(\delta_\nu)$ distributions are weighted to generic power law fluxes with indices $2$ and $2.5$ and to the current measurement from \cite{Haack:2017dxi} with a normalization constant of $\phi_0 = \SI[per-mode=reciprocal]{1}{\per\GeV\per\cm\squared\per\steradian\per\second}$.
    }
  \label{fig:effA_and_sindec}
\end{figure}


% %%%%%%%%%%%%%%%%%%%%%%%%%%%%%%%%%%%%%%%%%%%%%%%%%%%%%%%%%%%%%%%%%%%%%%%%%%%%%
% HESE decorrelation plots
% %%%%%%%%%%%%%%%%%%%%%%%%%%%%%%%%%%%%%%%%%%%%%%%%%%%%%%%%%%%%%%%%%%%%%%%%%%%%%
\begin{figure}[htbp]
  \centering
  \begin{subfigure}[t]{\textwidth}
    \centering
    \includegraphics{../datasets/plots/mc_no_hese_IC79.pdf}
    \subcaption{Sample IC79}
  \end{subfigure}
  \hfill
  \begin{subfigure}[t]{\textwidth}
    \centering
    \includegraphics{../datasets/plots/mc_no_hese_IC86_2011.pdf}
    \subcaption{Sample IC86, 2011}
  \end{subfigure}
  \hfill
  \begin{subfigure}[t]{\textwidth}
    \centering
    \includegraphics{../datasets/plots/mc_no_hese_IC86_2012-2014.pdf}
    \subcaption{Sample IC86, 2012--2014}
  \end{subfigure}
  \caption[HESE decorrelation for IC79, IC86'11, IC86'12--'14]{
    Plot showing the filtered out HESE-like events in the simulation files used for each sample on the right and the full simulation sample on the left.
    The samples are weighted to the flux model in \cite{Haack:2017dxi} equally normalized to a livetime of $365$ days.
    To obtain the total number of events, the integral over the parameter space needs to be taken, shown are differential number of events.
    }
  \label{fig:mc_no_hese_79_86I_86II}
\end{figure}
