\chapter{Appendix}
The appendix is mainly used to show additional plots, for example for distributions for other time windows than shown in the main matter.
These are often similar to each other and would otherwise clutter the main part too much.

\section{Further references}
The code written to enable the computational part of this analysis was mainly done in the \emph{Python} programming language, making heavy use of the scientific computing and visualization libraries \emph{numpy} \cite{numpy}, \emph{scipy} \cite{scipy}, \emph{matplotlib} \cite{matplotlib} and \emph{healpy} \cite{Gorski:2004by}.

\section{Acknowledgements}
\FIX{Missing personal acknowledgements}


\chapter{Datasets}
This appendix chapter includes additional plots for chapter~\ref{chp:datasets}.

\begin{figure}[htbp]
  \centering
  \includegraphics[width=0.9\textwidth]{../datasets/plots/hese_events_reco_landscape_skymap.pdf}
  \caption[Combined Likelihood skymap of the 22 HESEs]{
    Skymap in equatorial coordinates and Mollweide projection of the combination of all Likelihood reconstruction maps from the track-like high energy starting events.
  }
  \label{fig:hese_events_reco_landscape_skymap}
\end{figure}
