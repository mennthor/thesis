\chapter{Further references}
The appendix is mainly used to show additional plots, for example for distributions for other time windows than shown in the main matter.
These are often similar to each other and would otherwise clutter the main part too much.

The code written to enable the computational part of this analysis was mainly done in the \lstinline!Python! programming language, making heavy use of the scientific computing and visualization libraries \lstinline!numpy! \cite{numpy}, \lstinline!scipy! \cite{scipy}, \lstinline!matplotlib! \cite{matplotlib} and \lstinline!healpy! \cite{Gorski:2004by}.

\section{Notes on reproducibility}
The author tried to leave the analysis code in a state that hopefully allows to reproduce the distributions and results in this analysis.
\lstinline!Python! is used as the programming language for both the core analysis code and for the scripts using that core code.
The scripts and data for both analysis can be found on the computing cluster of the IceCube collaboration at the University of Wisconsin-Madison (UW-Madison).
A numbering scheme is used to guide the execution order of the scripts.
Help strings are included, when additional command line argument are needed.

The core analysis code for the time-dependent analysis was written from scratch, but it was tried to keep a close connection to the existing \lstinline!skylab! code base used for many other IceCube point-source searches.
At the time of writing this thesis, the \lstinline!skylab! module wasn't capable of handling the needed time-dependent variations of the extended Likelihood formalism though, making the self developed code necessary.
The code can be found in the \lstinline!github! repository at \url{www.github.com/mennthor/tdepps}.
This repository also includes spline and statistics related methods to provide the tools used during the evaluation of the analysis in the \lstinline!tdepps/utils! submodule.
The installation manual can be found in the modules \lstinline!Readme! file.
The scripts for the time-dependent analysis can be found on the UW-Madison cluster's file system under
\begin{lstlisting}
  /home/tmenne/analysis/hese_transient_stacking_analysis
\end{lstlisting}
An additional \lstinline!Readme! file is provided with a short install manual.
The analysis folder itself is a \lstinline!git! repository so all changes can be tracked.
For the first setup, the required Python dependencies can be installed using the \lstinline!pip! package manager and the provided \lstinline!py2_requirements.txt! file.
In this analysis the branch \lstinline!original_hese! is used.
The \lstinline!master! branch was intended to use an improved HESE sample event selection, \enquote{Pass2}, which was not ready at the time of this thesis and the original selection was used instead.
In general, the analysis folder uses \lstinline!git! branches and their names for the whole folder tree.
For this, a \lstinline!_loader.py! and \lstinline!_paths.py! module is provided.
These modules are used to handle all relative path dependencies and automatically set a new folder structure in the analysis folder and in the users data storage folder when a new branch is created.
This way, new tests cannot easily overwrite existing data or results and experiments can be done by \lstinline!git! branching.
The processing order loosely follows the steps presented in the analysis chapters~\ref{chp:time_dep}and~\ref{chp:time_indep}.
Some of the scripts simply run by themselves directly on the gateway machine.
However, trial generation takes quite some computing time.
The trial generation scripts are therefore usually split into three parts, where the first one creates job files for the \lstinline!dagman! distributed computing system, the second one resembles the actual code to be executed in each job and the last one is usually a script that collects and combines all the results from the job files in a more compact data structure.
For this analysis, all results are stored in a human readable form in the \lstinline!JSON! file format.
Because the format has no compression and can be quite verbose for a large datasets, the \lstinline!gzip! compression algorithm is used to compress the files if necessary.
The handling of these files in Python is quite easy as demonstrated in the following listing
\begin{lstlisting}
  import json
  import gzip
  fname = "path/to/file.json.gz"
  # 'gzip' may be optional, use just 'open(...)' then
  with gzip.open(fname, "r") as fp:
    data_dict = json.load(fp)
\end{lstlisting}

For the time-integrated search, the existing \lstinline!skylab! core analysis code was used.
Apart from that, the analysis is structured as explained above for the time-dependent analysis.
The Likelihood map signal injector was not available to the date when it was needed in the \lstinline!skylab! core code.
Due to the similar code structure between the self-developed \lstinline!tdepps! and the \lstinline!skylab! code, a simple adapter method was used to inject the signal events in a format needed by the \lstinline!skylab! code.
The analysis folder can also be found on the UW-Madison cluster file system at the path
\begin{lstlisting}
  /home/tmenne/analysis/hese_time_independent_stacking_fit_index
\end{lstlisting}
Unfortunately, the code to perform certain post-processing analysis steps, as the $\chi^2$ fits or code for producing the plots in this thesis completely resides in \lstinline!IPython! notebooks.
Although these have been copied to the respective analysis folders on the UW-Madison machine, this is not very beneficial for reproducibility.
The author can only recommend the reader to proceed otherwise with his analysis and make these accessible in standard scripts too.

\newpage
\section{Acknowledgements}
It would not have been possible to finish this thesis without the help and support of many people.
Trying to thank each and everyone of them would consequently lead to missing acknowledgement for sure.
Therefore, I only want to address some of them directly at this place, but also express my gratitude to everyone who participated in any way during the creation of this work and in general during my studies at TU Dortmund.

Firstly, I want to thank my parents for making it possible in the first place, with every small decision they made, for me to finish the physics education and the doctoral studies.
I also want to thank my wife for her support even though the writing of this thesis did take much longer than expected.

Furthermore, I want to thank my supervisor Dr.~Dr.~Wolfgang~Rhode for his professional and the financial support, as well as Dr.~Bernhardt~Spaan in his role as second corrector and the other two members of the defence committee Dr.~Bärbel~Siegmann and Dr.~Mirko~Cinchetti.

Also, particular thanks to the IceCube group at Drexel university, lead by Dr.Kurahashi~Neilson, where I had a great stay and learned a lot about the needed analysis techniques and to the DAAD for its financial support during the scholarship.

Lastly, many thanks go to my colleagues at the astroparticle physics chair at TU Dortmund, who helped me with many details and problems over the years and went through the pain of proof-reading this thesis.

\chapter{Supplementary material}
This chapter includes supplementary material, mostly additional figures, that would otherwise clutter the main chapters too much, but provide useful additional information.


\section{Datasets}
This appendix chapter includes additional figures for chapter~\ref{chp:datasets}.

% %%%%%%%%%%%%%%%%%%%%%%%%%%%%%%%%%%%%%%%%%%%%%%%%%%%%%%%%%%%%%%%%%%%%%%%%%%%%%
% HESE LLH maps skymap
% %%%%%%%%%%%%%%%%%%%%%%%%%%%%%%%%%%%%%%%%%%%%%%%%%%%%%%%%%%%%%%%%%%%%%%%%%%%%%
\begin{figure}[H]
  \centering
  \includegraphics{../datasets/plots/hese_events_reco_landscape_skymap.pdf}
  \caption[Combined Likelihood skymap of the 22 HES events]{
    Skymap in equatorial coordinates and Mollweide projection of the combination of all Likelihood reconstruction maps from the track-like high energy starting events.
  }
  \label{fig:hese_events_reco_landscape_skymap}
\end{figure}

% %%%%%%%%%%%%%%%%%%%%%%%%%%%%%%%%%%%%%%%%%%%%%%%%%%%%%%%%%%%%%%%%%%%%%%%%%%%%%
% Eff areas and sindec distributions
% %%%%%%%%%%%%%%%%%%%%%%%%%%%%%%%%%%%%%%%%%%%%%%%%%%%%%%%%%%%%%%%%%%%%%%%%%%%%%
\begin{figure}[H]
  \centering
  \includegraphics{../datasets/plots/effA_and_sindec.pdf}
  \caption[Effective areas and $\sin(\delta_\nu)$ distributions]{
    Effective areas and $\sin(\delta_\nu)$ distributions for for the samples IC79, IC86, 2011, IC86, 2012--2014 and IC86, 2015 in the muon neutrino test dataset.
    The effective areas are shown for multiple declination bands to capture the main features of the underlying sample selection efficiencies.
    The $\sin(\delta_\nu)$ distributions are weighted to generic power law fluxes with indices $2$ and $2.5$ and to the current measurement from \cite{Haack:2017dxi} all with the same normalisation constant of $\phi_0 = \SI[per-mode=reciprocal]{e-18}{\per\GeV\per\cm\squared\per\steradian\per\second}$.
    }
  \label{fig:effA_and_sindec}
\end{figure}

% %%%%%%%%%%%%%%%%%%%%%%%%%%%%%%%%%%%%%%%%%%%%%%%%%%%%%%%%%%%%%%%%%%%%%%%%%%%%%
% HESE decorrelation plots
% %%%%%%%%%%%%%%%%%%%%%%%%%%%%%%%%%%%%%%%%%%%%%%%%%%%%%%%%%%%%%%%%%%%%%%%%%%%%%
\begin{figure}[H]
  \centering
  \begin{subfigure}[t]{\textwidth}
    \centering
    \includegraphics{../datasets/plots/mc_no_hese_IC79.pdf}
    \subcaption{Sample IC79}
  \end{subfigure}
  \hfill
  \begin{subfigure}[t]{\textwidth}
    \centering
    \includegraphics{../datasets/plots/mc_no_hese_IC86_2011.pdf}
    \subcaption{Sample IC86, 2011}
  \end{subfigure}
  \hfill
  \begin{subfigure}[t]{\textwidth}
    \centering
    \includegraphics{../datasets/plots/mc_no_hese_IC86_2012-2014.pdf}
    \subcaption{Sample IC86, 2012--2014}
  \end{subfigure}
  \caption[HESE decorrelation for IC79, IC86'11 and IC86'12--'14]{
    Plot showing the filtered out HESE-like events in the simulation files used for each sample on the right and the full simulation sample on the left.
    The samples are weighted to the flux model in \cite{Haack:2017dxi} equally normalized to a livetime of $365$ days.
    To obtain the total number of events, the integral over the parameter space needs to be taken, shown are differential number of events.
    }
  \label{fig:mc_no_hese_79_86I_86II}
\end{figure}


\section{Time-dependent analysis}
This chapter includes additional plots for the time-dependent analysis chapter.

% %%%%%%%%%%%%%%%%%%%%%%%%%%%%%%%%%%%%%%%%%%%%%%%%%%%%%%%%%%%%%%%%%%%%%%%%%%%%%
% Sine rate model splines
% %%%%%%%%%%%%%%%%%%%%%%%%%%%%%%%%%%%%%%%%%%%%%%%%%%%%%%%%%%%%%%%%%%%%%%%%%%%%%
\enlargethispage*{5cm}
\begin{figure}[H]
  \centering
  \includegraphics{../time_dep_analysis/plots/sine_param_splines.pdf}
  \caption[Parameter splines for the sine rate model per sample]{
    Spline models fitted to the discrete fit parameters from the rate models in figures~(\ref{fig:rates_per_bin_IC79}--\ref{fig:rates_per_bin_IC86_2015}).
    The splines are used to obtain the set of rate model parameters used to tailor the model for the source locations in time.
    On the left the splines for the amplitude parameter $A$ and on right the splines for the baseline parameter $R_0$ for the model~(\ref{equ:rate_model}) are shown.
  }
  \label{fig:tdep_sine_param_splines}
\end{figure}

% %%%%%%%%%%%%%%%%%%%%%%%%%%%%%%%%%%%%%%%%%%%%%%%%%%%%%%%%%%%%%%%%%%%%%%%%%%%%%
% Rates for all samples
% %%%%%%%%%%%%%%%%%%%%%%%%%%%%%%%%%%%%%%%%%%%%%%%%%%%%%%%%%%%%%%%%%%%%%%%%%%%%%
\begin{figure}[H]
  \centering
  \includegraphics{../time_dep_analysis/plots/rates_all.pdf}
  \caption[Total run rates and sine models for all samples]{
    Measured run rates and fitted allsky rate models for all four muon track data samples.
    The rates per run are rebinned in monthly bins and a sine model is fitted to the bin centres per sample.
    This global model is integrated over each source's time window to estimate the expected number of background events, which is a priori fixed in the Likelihood fit.
  }
  \label{fig:rates_all}
\end{figure}
\enlargethispage*{5cm}
% %%%%%%%%%%%%%%%%%%%%%%%%%%%%%%%%%%%%%%%%%%%%%%%%%%%%%%%%%%%%%%%%%%%%%%%%%%%%%
% Stacking weights
% %%%%%%%%%%%%%%%%%%%%%%%%%%%%%%%%%%%%%%%%%%%%%%%%%%%%%%%%%%%%%%%%%%%%%%%%%%%%%
\begin{figure}[H]
  \centering
  \includegraphics{../time_dep_analysis/plots/stacking_src_w_per_sample.pdf}
  \caption[Source stacking weights for the time-dependent analysis]{
    Source stacking detection efficiency weights for all source per sample.
    The spline is constructed from the distribution of simulated signal-like events at the same processing level as the measured dataset.
    The weights per source are read of at the corresponding source declinations and are independently normalized per sample.
  }
  \label{fig:tdep_stacking_src_w_per_sample}
\end{figure}

% %%%%%%%%%%%%%%%%%%%%%%%%%%%%%%%%%%%%%%%%%%%%%%%%%%%%%%%%%%%%%%%%%%%%%%%%%%%%%
% Rates per bin plots
% %%%%%%%%%%%%%%%%%%%%%%%%%%%%%%%%%%%%%%%%%%%%%%%%%%%%%%%%%%%%%%%%%%%%%%%%%%%%%
\begin{figure}[H]
  \centering
  \includegraphics{../time_dep_analysis/plots/rates_per_bin_IC79.pdf}
  \caption[Rate models per bin for sample IC79]{
    Measured run rates and fitted models for the IC79 sample.
    In each bin, the rates per run are rebinned in monthly bins and a sine model is fitted to the bin centres.
  }
  \label{fig:rates_per_bin_IC79}
\end{figure}

\begin{figure}[H]
  \centering
  \includegraphics{../time_dep_analysis/plots/rates_per_bin_IC86_2011.pdf}
  \caption[Rate models per bin for sample IC86, 2011]{
    Measured run rates and fitted models for the IC86, 2011 sample.
    In each bin, the rates per run are rebinned in monthly bins and a sine model is fitted to the bin centres.
  }
  \label{fig:rates_per_bin_IC86_2011}
\end{figure}

\begin{figure}[H]
  \centering
  \includegraphics{../time_dep_analysis/plots/rates_per_bin_IC86_2012-2014.pdf}
  \caption[Rate models per bin for sample IC86, 2012--2014]{
    Measured run rates and fitted models for the IC86, 2012--2014 sample.
    In each bin, the rates per run are rebinned in monthly bins and a sine model is fitted to the bin centres.
  }
  \label{fig:rates_per_bin_IC86_2012-2014}
\end{figure}

\begin{figure}[H]
  \centering
  \includegraphics{../time_dep_analysis/plots/rates_per_bin_IC86_2015.pdf}
  \caption[Rate models per bin for sample IC86, 2015]{
    Measured run rates and fitted models for the IC86, 2015 sample.
    In each bin, the rates per run are rebinned in monthly bins and a sine model is fitted to the bin centres.
  }
  \label{fig:rates_per_bin_IC86_2015}
\end{figure}

% %%%%%%%%%%%%%%%%%%%%%%%%%%%%%%%%%%%%%%%%%%%%%%%%%%%%%%%%%%%%%%%%%%%%%%%%%%%%%
% KS test for independent LIDO trials
% %%%%%%%%%%%%%%%%%%%%%%%%%%%%%%%%%%%%%%%%%%%%%%%%%%%%%%%%%%%%%%%%%%%%%%%%%%%%%
\begin{figure}[H]
  \centering
  \includegraphics{../time_dep_analysis/plots/ks_test_lido_trials_pvals.pdf}
  \caption[KS test results for the independent BG trials]{
    Kolmogorov-Smirnov test results per time-window, testing the exponential tail for the background test statistics against a set of independently sampled distributions.
    A small tendency for a mismatch at larger time windows can be seen, but overall the fitted tails describe the sampled distributions sufficiently well.
  }
  \label{fig:ks_test_lido_trials_pvals}
\end{figure}
\enlargethispage*{5cm}
% %%%%%%%%%%%%%%%%%%%%%%%%%%%%%%%%%%%%%%%%%%%%%%%%%%%%%%%%%%%%%%%%%%%%%%%%%%%%%
% HESE map injector sampling
% %%%%%%%%%%%%%%%%%%%%%%%%%%%%%%%%%%%%%%%%%%%%%%%%%%%%%%%%%%%%%%%%%%%%%%%%%%%%%
\begin{figure}[H]
  \centering
  \includegraphics{../time_dep_analysis/plots/hese_map_sampling.pdf}
  \caption[Sampling from a healpy map for the signal source injection]{
    Likelihood map for the high energy starting event $1$ as used in the analysis.
    The dots are the sampled source positions from the signal injector module for $\num{e3}$ trials.
    Per trial one new source location is sampled per source.
    The actually injected signal events are rotated to the new source positions in their true coordinates prior to injection and added to the separately sampled background event sample.
  }
  \label{fig:hese_map_sampling}
\end{figure}

% %%%%%%%%%%%%%%%%%%%%%%%%%%%%%%%%%%%%%%%%%%%%%%%%%%%%%%%%%%%%%%%%%%%%%%%%%%%%%
% Performance chi2 fit per time window
% %%%%%%%%%%%%%%%%%%%%%%%%%%%%%%%%%%%%%%%%%%%%%%%%%%%%%%%%%%%%%%%%%%%%%%%%%%%%%
\begin{figure}[H]
  \centering
  \includegraphics{../time_dep_analysis/plots/perf_chi2_fits.pdf}
  \caption[$\chi^2$ CDF fits per time window for the analysis performance]{
     $\chi^2$ CDF fits to the discrete set of test statistic quantiles from the signal injection trials over the mean number of injected signal events $\mu$ parameter scan for all time windows.
     For each point, the quantile for the currently sampled test statistic over the desired test statistic value from the background distribution is calculated from the empiric PDF.
     The the $\chi^2$ CDF is fitted to all grid points to obtain a more reliable estimate of the needed quantile.
     The density of the sampled grid points in $\mu$ is adapted to the needed quantity in each bin.
     Each set of trials is reused for the sensitivity and discovery potential definitions.
     The injection is done with a unbroken power law with index $\gamma=2$ over the whole available simulation dataset true neutrino energy range.
  }
  \label{fig:tdep_perf_chi2_fits}
\end{figure}


% %%%%%%%%%%%%%%%%%%%%%%%%%%%%%%%%%%%%%%%%%%%%%%%%%%%%%%%%%%%%%%%%%%%%%%%%%%%%%
% BG test statistics. 0,1,2 and 19,20,21 are placed in the chapter
% %%%%%%%%%%%%%%%%%%%%%%%%%%%%%%%%%%%%%%%%%%%%%%%%%%%%%%%%%%%%%%%%%%%%%%%%%%%%%
\textbf{\Large\sffamily Background test statistics}
\begin{figure}[H]
  \centering
  \includegraphics{../time_dep_analysis/plots/bg_ts_and_ks_tw_ids_3_4_5.pdf}
  \caption[Background-only test statistics for the time windows 4, 5 and 6]{
    Background only test statistics for the time windows 4, 5 and 6 on the left and the parameter scan using the Kolmogorov-Smirnov (KS) test for the best threshold position on the right.
    The sampled PDF is described empirically up to the threshold value and with an exponential tail after that.
    The KS test is used to decide when the tail describes the sampled data sufficiently accurate.
  }
  \label{fig:bg_ts_and_ks_tw_ids_3_4_5}
\end{figure}

\begin{figure}[H]
  \centering
  \includegraphics{../time_dep_analysis/plots/bg_ts_and_ks_tw_ids_6_7_8.pdf}
  \caption[Background-only test statistics for the time windows 7, 8 and 9]{
    Background only test statistics for the time windows 7, 8 and 9 on the left and the parameter scan using the Kolmogorov-Smirnov (KS) test for the best threshold position on the right.
    The sampled PDF is described empirically up to the threshold value and with an exponential tail after that.
    The KS test is used to decide when the tail describes the sampled data sufficiently accurate.
  }
  \label{fig:bg_ts_and_ks_tw_ids_6_7_8}
\end{figure}

\begin{figure}[H]
  \centering
  \includegraphics{../time_dep_analysis/plots/bg_ts_and_ks_tw_ids_9_10_11.pdf}
  \caption[Background-only test statistics for the time windows 10, 11 and 12]{
    Background only test statistics for the time windows 10, 11 and 12 on the left and the parameter scan using the Kolmogorov-Smirnov (KS) test for the best threshold position on the right.
    The sampled PDF is described empirically up to the threshold value and with an exponential tail after that.
    The KS test is used to decide when the tail describes the sampled data sufficiently accurate.
  }
  \label{fig:bg_ts_and_ks_tw_ids_9_10_11}
\end{figure}

\begin{figure}[H]
  \centering
  \includegraphics{../time_dep_analysis/plots/bg_ts_and_ks_tw_ids_12_13_14.pdf}
  \caption[Background-only test statistics for the time windows 13, 14 and 15]{
    Background only test statistics for the time windows 13, 14 and 15 on the left and the parameter scan using the Kolmogorov-Smirnov (KS) test for the best threshold position on the right.
    The sampled PDF is described empirically up to the threshold value and with an exponential tail after that.
    The KS test is used to decide when the tail describes the sampled data sufficiently accurate.
  }
  \label{fig:bg_ts_and_ks_tw_ids_12_13_14}
\end{figure}

\begin{figure}[H]
  \centering
  \includegraphics{../time_dep_analysis/plots/bg_ts_and_ks_tw_ids_15_16_17.pdf}
  \caption[Background-only test statistics for the time windows 16, 17 and 18]{
    Background only test statistics for the time windows 16, 17 and 18 on the left and the parameter scan using the Kolmogorov-Smirnov (KS) test for the best threshold position on the right.
    The sampled PDF is described empirically up to the threshold value and with an exponential tail after that.
    The KS test is used to decide when the tail describes the sampled data sufficiently accurate.
  }
  \label{fig:bg_ts_and_ks_tw_ids_15_16_17}
\end{figure}

% %%%%%%%%%%%%%%%%%%%%%%%%%%%%%%%%%%%%%%%%%%%%%%%%%%%%%%%%%%%%%%%%%%%%%%%%%%%%%
% Differential perf chi2 fits and values per bin for each time window
% %%%%%%%%%%%%%%%%%%%%%%%%%%%%%%%%%%%%%%%%%%%%%%%%%%%%%%%%%%%%%%%%%%%%%%%%%%%%%
\newpage
\textbf{\Large\sffamily Differential performances $\chi^2$ CDF fits}
% 1st table with sensitivity values, logE <=4.5
\begin{table}[H]
\centering
\caption[Differential performance fluence values per time window -- Part 1]{
  $E^2$ weighted numerical values for the differential sensitivity fluence normalisations at $\SI{100}{\TeV}$ in $\si{\GeV\per\cm\squared}$.
  The values correspond to the differential performance curves shown in figure~(\ref{fig:tdep_diff_perf}).
  The field ID corresponds to the time window ID as defined in table~(\ref{tab:time_windows}).
  See table~(\ref{tab:tdep_diff_perf_tab2}) for the other energy bins.
  Note: the unweighted fluence values are obtained by dividing by $(\SI{100}{\TeV})^2$.
  }
\label{tab:tdep_diff_perf_tab1}
\begin{tabular}{
    S[table-format = 2.0]  % time window ID
    S[table-format = 1.3]  % Fluence value column
    S[table-format = 1.3]  % Fluence value column
    S[table-format = 1.3]  % Fluence value column
    S[table-format = 1.3]  % Fluence value column
    S[table-format = 1.3]  % Fluence value column
    S[table-format = 1.3]  % Fluence value column
    S[table-format = 1.3]  % Fluence value column
  }
  \toprule
  {ID} &
    \multicolumn{7}{c}
      {\footnotesize Left bin edge $\log_{10}(E_\nu / \si{\GeV})$} \\
  \cline{2-8}
     & 2.0   & 2.5   & 3.0   & 3.5   & 4.0   & 4.5   & 5.0   \\
  \midrule
   0 & 0.083 & 0.084 & 0.086 & 0.089 & 0.093 & 0.099 & 0.107 \\
   1 & 0.084 & 0.086 & 0.087 & 0.090 & 0.094 & 0.100 & 0.108 \\
   2 & 0.086 & 0.087 & 0.088 & 0.091 & 0.096 & 0.102 & 0.110 \\
   3 & 0.087 & 0.089 & 0.090 & 0.092 & 0.098 & 0.103 & 0.112 \\
   4 & 0.089 & 0.090 & 0.092 & 0.095 & 0.098 & 0.105 & 0.113 \\
   5 & 0.090 & 0.093 & 0.093 & 0.097 & 0.102 & 0.108 & 0.116 \\
   6 & 0.092 & 0.095 & 0.096 & 0.099 & 0.103 & 0.110 & 0.119 \\
   7 & 0.094 & 0.097 & 0.099 & 0.101 & 0.106 & 0.112 & 0.123 \\
   8 & 0.099 & 0.101 & 0.102 & 0.103 & 0.108 & 0.116 & 0.125 \\
   9 & 0.103 & 0.105 & 0.105 & 0.108 & 0.113 & 0.120 & 0.129 \\
  10 & 0.109 & 0.110 & 0.110 & 0.111 & 0.117 & 0.123 & 0.133 \\
  11 & 0.118 & 0.117 & 0.116 & 0.117 & 0.121 & 0.127 & 0.139 \\
  12 & 0.130 & 0.127 & 0.124 & 0.124 & 0.127 & 0.133 & 0.144 \\
  13 & 0.148 & 0.140 & 0.135 & 0.132 & 0.134 & 0.139 & 0.152 \\
  14 & 0.174 & 0.159 & 0.149 & 0.145 & 0.144 & 0.148 & 0.157 \\
  15 & 0.212 & 0.187 & 0.169 & 0.159 & 0.154 & 0.156 & 0.166 \\
  16 & 0.276 & 0.227 & 0.199 & 0.180 & 0.169 & 0.168 & 0.176 \\
  17 & 0.378 & 0.290 & 0.241 & 0.210 & 0.190 & 0.183 & 0.190 \\
  18 & 0.528 & 0.388 & 0.310 & 0.258 & 0.218 & 0.203 & 0.206 \\
  19 & 0.761 & 0.532 & 0.407 & 0.318 & 0.257 & 0.230 & 0.227 \\
  20 & 1.105 & 0.739 & 0.548 & 0.409 & 0.309 & 0.262 & 0.253 \\
  \bottomrule
\end{tabular}
\end{table}
% 2nd table with sensitivity values, logE >4.5
\begin{table}[H]
\centering
\caption[Differential performance fluence values per time window -- Part 2]{
  $E^2$ weighted numerical values for the differential sensitivity fluence normalisations at $\SI{100}{\TeV}$ in $\si{\GeV\per\cm\squared}$.
  The values correspond to the differential performance curves shown in figure~(\ref{fig:tdep_diff_perf}).
  The field ID corresponds to the time window ID as defined in table~(\ref{tab:time_windows}).
  See table~(\ref{tab:tdep_diff_perf_tab1}) for the other energy bins.
  Note: the unweighted fluence values are obtained by dividing by $(\SI{100}{\TeV})^2$.
  }
\label{tab:tdep_diff_perf_tab2}
\begin{tabular}{
    S[table-format = 2.0]  % time window ID
    S[table-format = 1.3]  % Fluence value column
    S[table-format = 1.3]  % Fluence value column
    S[table-format = 1.3]  % Fluence value column
    S[table-format = 1.3]  % Fluence value column
    S[table-format = 1.3]  % Fluence value column
    S[table-format = 1.3]  % Fluence value column
    S[table-format = 1.3]  % Fluence value column
  }
  \toprule
  {ID} &
    \multicolumn{7}{c}
      {\footnotesize Left bin edge $\log_{10}(E_\nu / \si{\GeV})$} \\
  \cline{2-8}
     & 5.5 & 6.0 & 6.5 & 7.0 & 7.5 & 8.0 & 8.5 \\
  \midrule
   0 & 0.116 & 0.124 & 0.127 & 0.128 & 0.129 & 0.129 & 0.130 \\
   1 & 0.117 & 0.125 & 0.129 & 0.130 & 0.131 & 0.131 & 0.132 \\
   2 & 0.120 & 0.128 & 0.131 & 0.132 & 0.133 & 0.132 & 0.134 \\
   3 & 0.123 & 0.129 & 0.133 & 0.135 & 0.135 & 0.136 & 0.136 \\
   4 & 0.125 & 0.133 & 0.136 & 0.137 & 0.138 & 0.138 & 0.138 \\
   5 & 0.126 & 0.136 & 0.140 & 0.140 & 0.140 & 0.141 & 0.140 \\
   6 & 0.131 & 0.140 & 0.143 & 0.143 & 0.143 & 0.142 & 0.143 \\
   7 & 0.134 & 0.143 & 0.145 & 0.147 & 0.147 & 0.146 & 0.146 \\
   8 & 0.139 & 0.147 & 0.151 & 0.150 & 0.151 & 0.149 & 0.150 \\
   9 & 0.144 & 0.153 & 0.155 & 0.154 & 0.154 & 0.154 & 0.153 \\
  10 & 0.149 & 0.160 & 0.161 & 0.160 & 0.160 & 0.158 & 0.158 \\
  11 & 0.154 & 0.166 & 0.167 & 0.166 & 0.163 & 0.162 & 0.162 \\
  12 & 0.162 & 0.174 & 0.175 & 0.171 & 0.170 & 0.168 & 0.167 \\
  13 & 0.168 & 0.183 & 0.183 & 0.179 & 0.175 & 0.173 & 0.173 \\
  14 & 0.178 & 0.192 & 0.192 & 0.187 & 0.183 & 0.180 & 0.179 \\
  15 & 0.187 & 0.202 & 0.202 & 0.196 & 0.190 & 0.186 & 0.184 \\
  16 & 0.197 & 0.215 & 0.215 & 0.206 & 0.199 & 0.196 & 0.192 \\
  17 & 0.213 & 0.230 & 0.229 & 0.220 & 0.210 & 0.206 & 0.204 \\
  18 & 0.230 & 0.250 & 0.248 & 0.239 & 0.225 & 0.219 & 0.217 \\
  19 & 0.248 & 0.270 & 0.270 & 0.259 & 0.246 & 0.239 & 0.236 \\
  20 & 0.273 & 0.293 & 0.297 & 0.285 & 0.268 & 0.261 & 0.259 \\
  \bottomrule
\end{tabular}
\end{table}

% 1st
\begin{figure}[H]
  \centering
  \includegraphics{../time_dep_analysis/plots/diff_perf_chi2_fits_tw_00.pdf}
  \caption[$\chi^2$ CDF fits for the 1st time window differential performance]{
     $\chi^2$ CDF fits to the discrete set of test statistic quantiles from the signal injection trials over the mean number of injected signal events $\mu$ parameter scan for the first time window.
     For each point, the quantile for the currently sampled test statistic over the desired test statistic value from the background distribution is calculated from the empiric PDF.
     The $\chi^2$ CDF is fitted to the points to obtain a more reliable estimate of the needed quantile.
     The density of the sampled grid point in $\mu$ is adapted to the needed quantity in each bin.
     Each set of trials is reused for the sensitivity and discovery potential definitions.
  }
  \label{fig:diff_perf_chi2_fits_tw_00}
\end{figure}
% 2nd
\begin{figure}[H]
  \centering
  \includegraphics{../time_dep_analysis/plots/diff_perf_chi2_fits_tw_01.pdf}
  \caption[$\chi^2$ CDF fits for the 2nd time window differential performance]{
     $\chi^2$ CDF fits to the discrete set of test statistic quantiles from the signal injection trials over the mean number of injected signal events $\mu$ parameter scan for the second time window.
     For each point, the quantile for the currently sampled test statistic over the desired test statistic value from the background distribution is calculated from the empiric PDF.
     The $\chi^2$ CDF is fitted to the points to obtain a more reliable estimate of the needed quantile.
     The density of the sampled grid point in $\mu$ is adapted to the needed quantity in each bin.
     Each set of trials is reused for the sensitivity and discovery potential definitions.
  }
  \label{fig:diff_perf_chi2_fits_tw_01}
\end{figure}
% 3rd
\begin{figure}[H]
  \centering
  \includegraphics{../time_dep_analysis/plots/diff_perf_chi2_fits_tw_02.pdf}
  \caption[$\chi^2$ CDF fits for the 3rd time window differential performance]{
     $\chi^2$ CDF fits to the discrete set of test statistic quantiles from the signal injection trials over the mean number of injected signal events $\mu$ parameter scan for the third time window.
     For each point, the quantile for the currently sampled test statistic over the desired test statistic value from the background distribution is calculated from the empiric PDF.
     The $\chi^2$ CDF is fitted to the points to obtain a more reliable estimate of the needed quantile.
     The density of the sampled grid point in $\mu$ is adapted to the needed quantity in each bin.
     Each set of trials is reused for the sensitivity and discovery potential definitions.
  }
  \label{fig:diff_perf_chi2_fits_tw_02}
\end{figure}
% 4th
\begin{figure}[H]
  \centering
  \includegraphics{../time_dep_analysis/plots/diff_perf_chi2_fits_tw_03.pdf}
  \caption[$\chi^2$ CDF fits for the 4th time window differential performance]{
     $\chi^2$ CDF fits to the discrete set of test statistic quantiles from the signal injection trials over the mean number of injected signal events $\mu$ parameter scan for the fourth time window.
     For each point, the quantile for the currently sampled test statistic over the desired test statistic value from the background distribution is calculated from the empiric PDF.
     The $\chi^2$ CDF is fitted to the points to obtain a more reliable estimate of the needed quantile.
     The density of the sampled grid point in $\mu$ is adapted to the needed quantity in each bin.
     Each set of trials is reused for the sensitivity and discovery potential definitions.
  }
  \label{fig:diff_perf_chi2_fits_tw_03}
\end{figure}
% 5th
\begin{figure}[H]
  \centering
  \includegraphics{../time_dep_analysis/plots/diff_perf_chi2_fits_tw_04.pdf}
  \caption[$\chi^2$ CDF fits for the 5th time window differential performance]{
     $\chi^2$ CDF fits to the discrete set of test statistic quantiles from the signal injection trials over the mean number of injected signal events $\mu$ parameter scan for the fifth time window.
     For each point, the quantile for the currently sampled test statistic over the desired test statistic value from the background distribution is calculated from the empiric PDF.
     The $\chi^2$ CDF is fitted to the points to obtain a more reliable estimate of the needed quantile.
     The density of the sampled grid point in $\mu$ is adapted to the needed quantity in each bin.
     Each set of trials is reused for the sensitivity and discovery potential definitions.
  }
  \label{fig:diff_perf_chi2_fits_tw_04}
\end{figure}
% 6th
\begin{figure}[H]
  \centering
  \includegraphics{../time_dep_analysis/plots/diff_perf_chi2_fits_tw_05.pdf}
  \caption[$\chi^2$ CDF fits for the 6th time window differential performance]{
     $\chi^2$ CDF fits to the discrete set of test statistic quantiles from the signal injection trials over the mean number of injected signal events $\mu$ parameter scan for the sixth time window.
     For each point, the quantile for the currently sampled test statistic over the desired test statistic value from the background distribution is calculated from the empiric PDF.
     The $\chi^2$ CDF is fitted to the points to obtain a more reliable estimate of the needed quantile.
     The density of the sampled grid point in $\mu$ is adapted to the needed quantity in each bin.
     Each set of trials is reused for the sensitivity and discovery potential definitions.
  }
  \label{fig:diff_perf_chi2_fits_tw_05}
\end{figure}
% 7th
\begin{figure}[H]
  \centering
  \includegraphics{../time_dep_analysis/plots/diff_perf_chi2_fits_tw_06.pdf}
  \caption[$\chi^2$ CDF fits for the 7th time window differential performance]{
     $\chi^2$ CDF fits to the discrete set of test statistic quantiles from the signal injection trials over the mean number of injected signal events $\mu$ parameter scan for the seventh time window.
     For each point, the quantile for the currently sampled test statistic over the desired test statistic value from the background distribution is calculated from the empiric PDF.
     The $\chi^2$ CDF is fitted to the points to obtain a more reliable estimate of the needed quantile.
     The density of the sampled grid point in $\mu$ is adapted to the needed quantity in each bin.
     Each set of trials is reused for the sensitivity and discovery potential definitions.
  }
  \label{fig:diff_perf_chi2_fits_tw_06}
\end{figure}
% 8th
\begin{figure}[H]
  \centering
  \includegraphics{../time_dep_analysis/plots/diff_perf_chi2_fits_tw_07.pdf}
  \caption[$\chi^2$ CDF fits for the 8th time window differential performance]{
     $\chi^2$ CDF fits to the discrete set of test statistic quantiles from the signal injection trials over the mean number of injected signal events $\mu$ parameter scan for the eighth time window.
     For each point, the quantile for the currently sampled test statistic over the desired test statistic value from the background distribution is calculated from the empiric PDF.
     The $\chi^2$ CDF is fitted to the points to obtain a more reliable estimate of the needed quantile.
     The density of the sampled grid point in $\mu$ is adapted to the needed quantity in each bin.
     Each set of trials is reused for the sensitivity and discovery potential definitions.
  }
  \label{fig:diff_perf_chi2_fits_tw_07}
\end{figure}
% 9th
\begin{figure}[H]
  \centering
  \includegraphics{../time_dep_analysis/plots/diff_perf_chi2_fits_tw_08.pdf}
  \caption[$\chi^2$ CDF fits for the 9th time window differential performance]{
     $\chi^2$ CDF fits to the discrete set of test statistic quantiles from the signal injection trials over the mean number of injected signal events $\mu$ parameter scan for the ninth time window.
     For each point, the quantile for the currently sampled test statistic over the desired test statistic value from the background distribution is calculated from the empiric PDF.
     The $\chi^2$ CDF is fitted to the points to obtain a more reliable estimate of the needed quantile.
     The density of the sampled grid point in $\mu$ is adapted to the needed quantity in each bin.
     Each set of trials is reused for the sensitivity and discovery potential definitions.
  }
  \label{fig:diff_perf_chi2_fits_tw_08}
\end{figure}
% 10th
\begin{figure}[H]
  \centering
  \includegraphics{../time_dep_analysis/plots/diff_perf_chi2_fits_tw_09.pdf}
  \caption[$\chi^2$ CDF fits for the 10th time window differential performance]{
     $\chi^2$ CDF fits to the discrete set of test statistic quantiles from the signal injection trials over the mean number of injected signal events $\mu$ parameter scan for the tenth time window.
     For each point, the quantile for the currently sampled test statistic over the desired test statistic value from the background distribution is calculated from the empiric PDF.
     The $\chi^2$ CDF is fitted to the points to obtain a more reliable estimate of the needed quantile.
     The density of the sampled grid point in $\mu$ is adapted to the needed quantity in each bin.
     Each set of trials is reused for the sensitivity and discovery potential definitions.
  }
  \label{fig:diff_perf_chi2_fits_tw_09}
\end{figure}
% 11th
\begin{figure}[H]
  \centering
  \includegraphics{../time_dep_analysis/plots/diff_perf_chi2_fits_tw_10.pdf}
  \caption[$\chi^2$ CDF fits for the 11th time window differential performance]{
     $\chi^2$ CDF fits to the discrete set of test statistic quantiles from the signal injection trials over the mean number of injected signal events $\mu$ parameter scan for the 11th time window.
     For each point, the quantile for the currently sampled test statistic over the desired test statistic value from the background distribution is calculated from the empiric PDF.
     The $\chi^2$ CDF is fitted to the points to obtain a more reliable estimate of the needed quantile.
     The density of the sampled grid point in $\mu$ is adapted to the needed quantity in each bin.
     Each set of trials is reused for the sensitivity and discovery potential definitions.
  }
  \label{fig:diff_perf_chi2_fits_tw_10}
\end{figure}
% 12th
\begin{figure}[H]
  \centering
  \includegraphics{../time_dep_analysis/plots/diff_perf_chi2_fits_tw_11.pdf}
  \caption[$\chi^2$ CDF fits for the 12th time window differential performance]{
     $\chi^2$ CDF fits to the discrete set of test statistic quantiles from the signal injection trials over the mean number of injected signal events $\mu$ parameter scan for the 12th time window.
     For each point, the quantile for the currently sampled test statistic over the desired test statistic value from the background distribution is calculated from the empiric PDF.
     The $\chi^2$ CDF is fitted to the points to obtain a more reliable estimate of the needed quantile.
     The density of the sampled grid point in $\mu$ is adapted to the needed quantity in each bin.
     Each set of trials is reused for the sensitivity and discovery potential definitions.
  }
  \label{fig:diff_perf_chi2_fits_tw_11}
\end{figure}
% 13th
\begin{figure}[H]
  \centering
  \includegraphics{../time_dep_analysis/plots/diff_perf_chi2_fits_tw_12.pdf}
  \caption[$\chi^2$ CDF fits for the 13th time window differential performance]{
     $\chi^2$ CDF fits to the discrete set of test statistic quantiles from the signal injection trials over the mean number of injected signal events $\mu$ parameter scan for the 13th time window.
     For each point, the quantile for the currently sampled test statistic over the desired test statistic value from the background distribution is calculated from the empiric PDF.
     The $\chi^2$ CDF is fitted to the points to obtain a more reliable estimate of the needed quantile.
     The density of the sampled grid point in $\mu$ is adapted to the needed quantity in each bin.
     Each set of trials is reused for the sensitivity and discovery potential definitions.
  }
  \label{fig:diff_perf_chi2_fits_tw_12}
\end{figure}
% 14th
\begin{figure}[H]
  \centering
  \includegraphics{../time_dep_analysis/plots/diff_perf_chi2_fits_tw_13.pdf}
  \caption[$\chi^2$ CDF fits for the 14th time window differential performance]{
     $\chi^2$ CDF fits to the discrete set of test statistic quantiles from the signal injection trials over the mean number of injected signal events $\mu$ parameter scan for the 14th time window.
     For each point, the quantile for the currently sampled test statistic over the desired test statistic value from the background distribution is calculated from the empiric PDF.
     The $\chi^2$ CDF is fitted to the points to obtain a more reliable estimate of the needed quantile.
     The density of the sampled grid point in $\mu$ is adapted to the needed quantity in each bin.
     Each set of trials is reused for the sensitivity and discovery potential definitions.
  }
  \label{fig:diff_perf_chi2_fits_tw_13}
\end{figure}
% 15th
\begin{figure}[H]
  \centering
  \includegraphics{../time_dep_analysis/plots/diff_perf_chi2_fits_tw_14.pdf}
  \caption[$\chi^2$ CDF fits for the 15th time window differential performance]{
     $\chi^2$ CDF fits to the discrete set of test statistic quantiles from the signal injection trials over the mean number of injected signal events $\mu$ parameter scan for the 15th time window.
     For each point, the quantile for the currently sampled test statistic over the desired test statistic value from the background distribution is calculated from the empiric PDF.
     The $\chi^2$ CDF is fitted to the points to obtain a more reliable estimate of the needed quantile.
     The density of the sampled grid point in $\mu$ is adapted to the needed quantity in each bin.
     Each set of trials is reused for the sensitivity and discovery potential definitions.
  }
  \label{fig:diff_perf_chi2_fits_tw_14}
\end{figure}
% 16th
\begin{figure}[H]
  \centering
  \includegraphics{../time_dep_analysis/plots/diff_perf_chi2_fits_tw_15.pdf}
  \caption[$\chi^2$ CDF fits for the 16th time window differential performance]{
     $\chi^2$ CDF fits to the discrete set of test statistic quantiles from the signal injection trials over the mean number of injected signal events $\mu$ parameter scan for the 16th time window.
     For each point, the quantile for the currently sampled test statistic over the desired test statistic value from the background distribution is calculated from the empiric PDF.
     The $\chi^2$ CDF is fitted to the points to obtain a more reliable estimate of the needed quantile.
     The density of the sampled grid point in $\mu$ is adapted to the needed quantity in each bin.
     Each set of trials is reused for the sensitivity and discovery potential definitions.
  }
  \label{fig:diff_perf_chi2_fits_tw_15}
\end{figure}
% 17th
\begin{figure}[H]
  \centering
  \includegraphics{../time_dep_analysis/plots/diff_perf_chi2_fits_tw_16.pdf}
  \caption[$\chi^2$ CDF fits for the 17th time window differential performance]{
     $\chi^2$ CDF fits to the discrete set of test statistic quantiles from the signal injection trials over the mean number of injected signal events $\mu$ parameter scan for the 17th time window.
     For each point, the quantile for the currently sampled test statistic over the desired test statistic value from the background distribution is calculated from the empiric PDF.
     The $\chi^2$ CDF is fitted to the points to obtain a more reliable estimate of the needed quantile.
     The density of the sampled grid point in $\mu$ is adapted to the needed quantity in each bin.
     Each set of trials is reused for the sensitivity and discovery potential definitions.
  }
  \label{fig:diff_perf_chi2_fits_tw_16}
\end{figure}
% 18th
\begin{figure}[H]
  \centering
  \includegraphics{../time_dep_analysis/plots/diff_perf_chi2_fits_tw_17.pdf}
  \caption[$\chi^2$ CDF fits for the 18th time window differential performance]{
     $\chi^2$ CDF fits to the discrete set of test statistic quantiles from the signal injection trials over the mean number of injected signal events $\mu$ parameter scan for the 18th time window.
     For each point, the quantile for the currently sampled test statistic over the desired test statistic value from the background distribution is calculated from the empiric PDF.
     The $\chi^2$ CDF is fitted to the points to obtain a more reliable estimate of the needed quantile.
     The density of the sampled grid point in $\mu$ is adapted to the needed quantity in each bin.
     Each set of trials is reused for the sensitivity and discovery potential definitions.
  }
  \label{fig:diff_perf_chi2_fits_tw_17}
\end{figure}
% 19th
\begin{figure}[H]
  \centering
  \includegraphics{../time_dep_analysis/plots/diff_perf_chi2_fits_tw_18.pdf}
  \caption[$\chi^2$ CDF fits for the 19th time window differential performance]{
     $\chi^2$ CDF fits to the discrete set of test statistic quantiles from the signal injection trials over the mean number of injected signal events $\mu$ parameter scan for the 19th time window.
     For each point, the quantile for the currently sampled test statistic over the desired test statistic value from the background distribution is calculated from the empiric PDF.
     The $\chi^2$ CDF is fitted to the points to obtain a more reliable estimate of the needed quantile.
     The density of the sampled grid point in $\mu$ is adapted to the needed quantity in each bin.
     Each set of trials is reused for the sensitivity and discovery potential definitions.
  }
  \label{fig:diff_perf_chi2_fits_tw_18}
\end{figure}
% 20th
\begin{figure}[H]
  \centering
  \includegraphics{../time_dep_analysis/plots/diff_perf_chi2_fits_tw_19.pdf}
  \caption[$\chi^2$ CDF fits for the 20th time window differential performance]{
     $\chi^2$ CDF fits to the discrete set of test statistic quantiles from the signal injection trials over the mean number of injected signal events $\mu$ parameter scan for the 20th time window.
     For each point, the quantile for the currently sampled test statistic over the desired test statistic value from the background distribution is calculated from the empiric PDF.
     The $\chi^2$ CDF is fitted to the points to obtain a more reliable estimate of the needed quantile.
     The density of the sampled grid point in $\mu$ is adapted to the needed quantity in each bin.
     Each set of trials is reused for the sensitivity and discovery potential definitions.
  }
  \label{fig:diff_perf_chi2_fits_tw_19}
\end{figure}
% 21st
\begin{figure}[H]
  \centering
  \includegraphics{../time_dep_analysis/plots/diff_perf_chi2_fits_tw_20.pdf}
  \caption[$\chi^2$ CDF fits for the 21st time window differential performance]{
     $\chi^2$ CDF fits to the discrete set of test statistic quantiles from the signal injection trials over the mean number of injected signal events $\mu$ parameter scan for the 21st time window.
     For each point, the quantile for the currently sampled test statistic over the desired test statistic value from the background distribution is calculated from the empiric PDF.
     The $\chi^2$ CDF is fitted to the points to obtain a more reliable estimate of the needed quantile.
     The density of the sampled grid point in $\mu$ is adapted to the needed quantity in each bin.
     Each set of trials is reused for the sensitivity and discovery potential definitions.
  }
  \label{fig:diff_perf_chi2_fits_tw_20}
\end{figure}


\section{Time-integrated analysis}
This chapter includes additional plots for the time-integrated analysis chapter.
% %%%%%%%%%%%%%%%%%%%%%%%%%%%%%%%%%%%%%%%%%%%%%%%%%%%%%%%%%%%%%%%%%%%%%%%%%%%%%
% Sample split weights dependent on spectral index
% %%%%%%%%%%%%%%%%%%%%%%%%%%%%%%%%%%%%%%%%%%%%%%%%%%%%%%%%%%%%%%%%%%%%%%%%%%%%%
\begin{figure}[H]
  \centering
  \includegraphics{../time_int_analysis/plots/sample_split_weights.pdf}
  \caption[Signal split weights for the time-integrated analysis]{
    Weights $w_j$ to split the global $n_S$ fit parameter across the single Likelihoods in the multi-sample Likelihood~(\ref{equ:multi_sam_time_indep}).
    The weights depend on the spectral index $\gamma$ that currently describes the signal flux hypothesis and are selected accordingly during the fitting procedure.
  }
  \label{fig:sample_split_weights}
\end{figure}
\enlargethispage*{5cm}
% %%%%%%%%%%%%%%%%%%%%%%%%%%%%%%%%%%%%%%%%%%%%%%%%%%%%%%%%%%%%%%%%%%%%%%%%%%%%%
% Injected events for differential performances
% %%%%%%%%%%%%%%%%%%%%%%%%%%%%%%%%%%%%%%%%%%%%%%%%%%%%%%%%%%%%%%%%%%%%%%%%%%%%%
\begin{figure}[H]
  \centering
  \includegraphics{../time_int_analysis/plots/diff_perf_inj_mu_comp.pdf}
  \caption[Injected signal events for the time-integrated diff. performance fluxes]{
    The number of needed injected signal events is almost identical for both injection models and for sensitivity and discovery potential fluxes due to the used small bin sizes.
    This quasi independence on the injection model is one prerequisite to justify the calculation of global flux performances or limits of almost arbitrary flux models from the differential fluxes.
  }
  \label{fig:tindep_diff_perf_inj_mu_comp}
\end{figure}


% %%%%%%%%%%%%%%%%%%%%%%%%%%%%%%%%%%%%%%%%%%%%%%%%%%%%%%%%%%%%%%%%%%%%%%%%%%%%%
% Energy PDFs per gamma and per sample, IC79 in the main text
% %%%%%%%%%%%%%%%%%%%%%%%%%%%%%%%%%%%%%%%%%%%%%%%%%%%%%%%%%%%%%%%%%%%%%%%%%%%%%
\begin{figure}[H]
  \centering
  \includegraphics{../time_int_analysis/plots/energy_sob_IC86_2011.pdf}
  \caption[Energy PDF for the time-integrated analysis for IC86, 2011]{
    Two-dimensional ratio of the signal and background energy PDFs in $\log_{10}\left(E_\text{proxy}\right)$ and $\sin(\delta)$ for the IC86, 2011 sample and for different spectral indices $\gamma$.
    The underlying binning is the same for the signal and background histogram.
    Each ratio is pre-calculated for a grid of spectral indices beforehand and used during the fitting procedure for the current realization of $\gamma$.
    A one-dimensional spline, which is not shown here, is used per $\left(\sin(\delta), \log_{10}\left(E_\text{proxy}\right)\right)$ tuple for each event to interpolate the grid and obtain the gradient information for the spectral index fit parameter.
  }
  \label{fig:tindep_energy_sob_IC86_2011}
\end{figure}

\begin{figure}[H]
  \centering
  \includegraphics{../time_int_analysis/plots/energy_sob_IC86_2012-2014.pdf}
  \caption[Energy PDF for the time-integrated analysis for IC86, 2012--2014]{
    Two-dimensional ratio of the signal and background energy PDFs in $\log_{10}\left(E_\text{proxy}\right)$ and $\sin(\delta)$ for the IC86, 2012--2014 sample and for different spectral indices $\gamma$.
    The underlying binning is the same for the signal and background histogram.
    Each ratio is pre-calculated for a grid of spectral indices beforehand and used during the fitting procedure for the current realization of $\gamma$.
    A one-dimensional spline, which is not shown here, is used per $\left(\sin(\delta), \log_{10}\left(E_\text{proxy}\right)\right)$ tuple for each event to interpolate the grid and obtain the gradient information for the spectral index fit parameter.
  }
  \label{fig:tindep_energy_sob_IC86_2012-2014}
\end{figure}

\begin{figure}[H]
  \centering
  \includegraphics{../time_int_analysis/plots/energy_sob_IC86_2015.pdf}
  \caption[Energy PDF for the time-integrated analysis for IC86, 2015]{
    Two-dimensional ratio of the signal and background energy PDFs in $\log_{10}\left(E_\text{proxy}\right)$ and $\sin(\delta)$ for the IC86, 2015 sample and for different spectral indices $\gamma$.
    The underlying binning is the same for the signal and background histogram.
    Each ratio is pre-calculated for a grid of spectral indices beforehand and used during the fitting procedure for the current realization of $\gamma$.
    A one-dimensional spline, which is not shown here, is used per $\left(\sin(\delta), \log_{10}\left(E_\text{proxy}\right)\right)$ tuple for each event to interpolate the grid and obtain the gradient information for the spectral index fit parameter.
  }
  \label{fig:tindep_energy_sob_IC86_2015}
\end{figure}

% %%%%%%%%%%%%%%%%%%%%%%%%%%%%%%%%%%%%%%%%%%%%%%%%%%%%%%%%%%%%%%%%%%%%%%%%%%%%%
% BG TS with both models
% %%%%%%%%%%%%%%%%%%%%%%%%%%%%%%%%%%%%%%%%%%%%%%%%%%%%%%%%%%%%%%%%%%%%%%%%%%%%%
\begin{figure}[H]
  \centering
  \includegraphics{../time_int_analysis/plots/bg_ts_with_models_and_ks.pdf}
  \caption[Background test statistic for the time-integrated analysis]{
    Background test statistic for the time-integrated analysis with both the direct $\chi^2$ model and the hybrid model used in the time-dependent analysis for comparison.
    Only the effective degrees of freedom are fitted in the $\chi^2$ model, the number of non-zero trials $\eta$ is obtained by simple counting.
    On the right, the threshold scan for the hybrid model is shown.
    For the final p-value estimation, the $\chi^2$ model is used.
  }
  \label{fig:bg_ts_with_models_and_ks}
\end{figure}

% %%%%%%%%%%%%%%%%%%%%%%%%%%%%%%%%%%%%%%%%%%%%%%%%%%%%%%%%%%%%%%%%%%%%%%%%%%%%%
% BG ns and gamma hist
% %%%%%%%%%%%%%%%%%%%%%%%%%%%%%%%%%%%%%%%%%%%%%%%%%%%%%%%%%%%%%%%%%%%%%%%%%%%%%
\begin{figure}[H]
  \centering
  \includegraphics{../time_int_analysis/plots/bg_ts_ns_and_gamma_hist.pdf}
  \caption[Signal strength and spectral index distribution for background trials]{
    On the left plot, the $n_S$ distribution for pure background trials is shown.
    On the right plot, the same for the second fit parameter, the spectral index $\gamma$ of the assumed signal flux.
    A slight bias can be seen towards slightly harder spectra than expected from the common atmospheric neutrino flux model.
    However, the spectral index is handled more than a nuisance parameter in this type of analysis, so a slight bias has no negative consequence on the final result.
  }
  \label{fig:bg_ts_ns_and_gamma_hist}
\end{figure}

% %%%%%%%%%%%%%%%%%%%%%%%%%%%%%%%%%%%%%%%%%%%%%%%%%%%%%%%%%%%%%%%%%%%%%%%%%%%%%
% Differential perf chi2 fits and values per bin per gamma
% %%%%%%%%%%%%%%%%%%%%%%%%%%%%%%%%%%%%%%%%%%%%%%%%%%%%%%%%%%%%%%%%%%%%%%%%%%%%%
\newpage
\textbf{\Large\sffamily Differential performances $\chi^2$ CDF fits}
\enlargethispage*{5cm}
\begin{figure}[H]
  \centering
  \includegraphics{../time_int_analysis/plots/diff_perf_chi2_fits_gamma_2_chunk0.pdf}
  \caption[$\chi^2$ CDF fits for the time-integrated diff. performance, $\gamma_\text{inj}=2$ -- part 1]{
     $\chi^2$ CDF fits to the discrete set of test statistic quantiles from the signal injection trials over the mean number of injected signal events $\mu$ parameter scan for the injection model with spectral index $\gamma_\text{inj}=2$ and the lower half of neutrino energy bins.
     For each point, the quantile for the currently sampled test statistic over the desired test statistic value from the background distribution is calculated from the empiric PDF.
     The $\chi^2$ CDF is fitted to the points to obtain a more reliable estimate of the needed quantile.
     The density of the sampled grid point in $\mu$ is adapted to the needed quantity in each bin.
     The fits for the other half of the bins can be seen in figure~(\ref{fig:tindep_diff_perf_chi2_fits_gamma_2_chunk1})
  }
  \label{fig:tindep_diff_perf_chi2_fits_gamma_2_chunk0}
\end{figure}

\begin{figure}[H]
  \centering
  \includegraphics{../time_int_analysis/plots/diff_perf_chi2_fits_gamma_2_chunk1.pdf}
  \caption[$\chi^2$ CDF fits for the time-integrated diff. performance, $\gamma_\text{inj}=2$ -- part 2]{
     $\chi^2$ CDF fits to the discrete set of test statistic quantiles from the signal injection trials over the mean number of injected signal events $\mu$ parameter scan for the injection model with spectral index $\gamma_\text{inj}=2$ and the upper half of neutrino energy bins.
     For each point, the quantile for the currently sampled test statistic over the desired test statistic value from the background distribution is calculated from the empiric PDF.
     The $\chi^2$ CDF is fitted to the points to obtain a more reliable estimate of the needed quantile.
     The density of the sampled grid point in $\mu$ is adapted to the needed quantity in each bin.
     The fits for the other half of the bins can be seen in figure~(\ref{fig:tindep_diff_perf_chi2_fits_gamma_2_chunk0})
  }
  \label{fig:tindep_diff_perf_chi2_fits_gamma_2_chunk1}
\end{figure}

\begin{figure}[H]
  \centering
  \includegraphics{../time_int_analysis/plots/diff_perf_chi2_fits_gamma_3_chunk0.pdf}
  \caption[$\chi^2$ CDF fits for the time-integrated diff. performance, $\gamma_\text{inj}=3$ -- part 1]{
     $\chi^2$ CDF fits to the discrete set of test statistic quantiles from the signal injection trials over the mean number of injected signal events $\mu$ parameter scan for the injection model with spectral index $\gamma_\text{inj}=3$ and the lower half of neutrino energy bins.
     For each point, the quantile for the currently sampled test statistic over the desired test statistic value from the background distribution is calculated from the empiric PDF.
     The $\chi^2$ CDF is fitted to the points to obtain a more reliable estimate of the needed quantile.
     The density of the sampled grid point in $\mu$ is adapted to the needed quantity in each bin.
     The fits for the other half of the bins can be seen in figure~(\ref{fig:tindep_diff_perf_chi2_fits_gamma_3_chunk1})
  }
  \label{fig:tindep_diff_perf_chi2_fits_gamma_3_chunk0}
\end{figure}

\begin{figure}[H]
  \centering
  \includegraphics{../time_int_analysis/plots/diff_perf_chi2_fits_gamma_3_chunk1.pdf}
  \caption[$\chi^2$ CDF fits for the time-integrated diff. performance, $\gamma_\text{inj}=3$ -- part 2]{
     $\chi^2$ CDF fits to the discrete set of test statistic quantiles from the signal injection trials over the mean number of injected signal events $\mu$ parameter scan for the injection model with spectral index $\gamma_\text{inj}=3$ and the upper half of neutrino energy bins.
     For each point, the quantile for the currently sampled test statistic over the desired test statistic value from the background distribution is calculated from the empiric PDF.
     The $\chi^2$ CDF is fitted to the points to obtain a more reliable estimate of the needed quantile.
     The density of the sampled grid point in $\mu$ is adapted to the needed quantity in each bin.
     The fits for the other half of the bins can be seen in figure~(\ref{fig:tindep_diff_perf_chi2_fits_gamma_3_chunk0})
  }
  \label{fig:tindep_diff_perf_chi2_fits_gamma_3_chunk1}
\end{figure}

% 1st table with sensitivity values, gamma=2
\begin{table}[H]
\centering
\caption[Time-integrated diff. performance flux values per bin, $\gamma_\text{inj}=2$]{
  Numerical values for the differential sensitivity flux normalisations $E^2\phi_0^{\SI{100}{\TeV}}$ at $\SI{100}{\TeV}$ in $\si[per-mode=reciprocal]{\GeV\per\cm\squared\per\second}$.
  The values correspond to the differential performance curve shown in figure~(\ref{fig:tindep_diff_perf}) for the injection model with spectral index $\gamma_\text{inj}=2$.
  The $\log_{10}(E_\nu / \si{\GeV})$ columns are the left energy bin borders.
  See table~(\ref{tab:tindep_diff_perf_gamma3}) for the flux values calculated with the injection index $\gamma_\text{inj}=3$.
  Note: the unweighted flux values are obtained by dividing by $(\SI{100}{\TeV})^2$.
  }
\label{tab:tindep_diff_perf_gamma2}
\begin{tabular}{
    % https://tex.stackexchange.com/questions/132763
    S[table-format = 1.3]  % logE value column
    S[table-format = 1.2e+1]  % Flux value column
    S[table-format = 1.3]  % logE value column
    S[table-format = 1.2e+1]  % Flux value column
  }
  \toprule
    {$\log_{10}(E_\nu / \si{\GeV})$} & {$\phi_0^{\SI{100}{\TeV}}$} &
    {$\log_{10}(E_\nu / \si{\GeV})$} & {$\phi_0^{\SI{100}{\TeV}}$} \\
  \midrule
    2.000 & 5.34e-4 & 5.500 & 2.12e-7 \\
    2.125 & 1.45e-4 & 5.625 & 2.14e-7 \\
    2.250 & 5.41e-5 & 5.750 & 2.24e-7 \\
    2.375 & 2.52e-5 & 5.875 & 2.44e-7 \\
    2.500 & 1.36e-5 & 6.000 & 2.62e-7 \\
    2.625 & 8.35e-6 & 6.125 & 2.83e-7 \\
    2.750 & 5.55e-6 & 6.250 & 3.31e-7 \\
    2.875 & 3.80e-6 & 6.375 & 3.61e-7 \\
    3.000 & 2.75e-6 & 6.500 & 4.15e-7 \\
    3.125 & 2.09e-6 & 6.625 & 4.50e-7 \\
    3.250 & 1.64e-6 & 6.750 & 5.15e-7 \\
    3.375 & 1.31e-6 & 6.875 & 5.59e-7 \\
    3.500 & 1.09e-6 & 7.000 & 6.50e-7 \\
    3.625 & 9.13e-7 & 7.125 & 7.29e-7 \\
    3.750 & 7.93e-7 & 7.250 & 8.30e-7 \\
    3.875 & 6.85e-7 & 7.375 & 8.55e-7 \\
    4.000 & 5.89e-7 & 7.500 & 1.08e-6 \\
    4.125 & 5.11e-7 & 7.625 & 1.23e-6 \\
    4.250 & 4.46e-7 & 7.750 & 1.44e-6 \\
    4.375 & 3.92e-7 & 7.875 & 1.67e-6 \\
    4.500 & 3.48e-7 & 8.000 & 2.08e-6 \\
    4.625 & 3.17e-7 & 8.125 & 2.25e-6 \\
    4.750 & 2.78e-7 & 8.250 & 2.86e-6 \\
    4.875 & 2.56e-7 & 8.375 & 3.65e-6 \\
    5.000 & 2.36e-7 & 8.500 & 4.48e-6 \\
    5.125 & 2.11e-7 & 8.625 & 5.59e-6 \\
    5.250 & 2.07e-7 & 8.750 & 6.99e-6 \\
    5.375 & 2.12e-7 & 8.875 & 8.75e-6 \\
  \bottomrule
\end{tabular}
\end{table}
% 2nd table with sensitivity values, gamma=3
\begin{table}[H]
\centering
\caption[Time-integrated diff. performance flux values per bin, $\gamma_\text{inj}=3$]{
  Numerical values for the differential sensitivity flux normalisations $E^2\phi_0^{\SI{100}{\TeV}}$ at $\SI{100}{\TeV}$ in $\si[per-mode=reciprocal]{\GeV\per\cm\squared\per\second}$.
  The values correspond to the differential performance curve shown in figure~(\ref{fig:tindep_diff_perf}) for the injection model with spectral index $\gamma_\text{inj}=3$.
  The $\log_{10}(E_\nu / \si{\GeV})$ columns are the left energy bin borders.
  See table~(\ref{tab:tindep_diff_perf_gamma2}) for the flux values calculated with the injection index $\gamma_\text{inj}=2$.
  Note: the unweighted flux values are obtained by dividing by $(\SI{100}{\TeV})^2$.
  }
\label{tab:tindep_diff_perf_gamma3}
\begin{tabular}{
    S[table-format = 1.3]  % logE value column
    S[table-format = 1.2e+1]  % Flux value column
    S[table-format = 1.3]  % logE value column
    S[table-format = 1.2e+1]  % Flux value column
  }
  \toprule
    {$\log_{10}(E_\nu / \si{\GeV})$} & {$\phi_0^{\SI{100}{\TeV}}$} &
    {$\log_{10}(E_\nu / \si{\GeV})$} & {$\phi_0^{\SI{100}{\TeV}}$} \\
  \midrule
    2.000 & 6.37e-7 & 5.500 & 7.49e-7 \\
    2.125 & 2.27e-7 & 5.625 & 1.04e-6 \\
    2.250 & 1.12e-7 & 5.750 & 1.48e-6 \\
    2.375 & 7.01e-8 & 5.875 & 2.08e-6 \\
    2.500 & 5.01e-8 & 6.000 & 3.04e-6 \\
    2.625 & 4.13e-8 & 6.125 & 4.46e-6 \\
    2.750 & 3.65e-8 & 6.250 & 6.91e-6 \\
    2.875 & 3.35e-8 & 6.375 & 9.77e-6 \\
    3.000 & 3.19e-8 & 6.500 & 1.49e-5 \\
    3.125 & 3.29e-8 & 6.625 & 2.18e-5 \\
    3.250 & 3.35e-8 & 6.750 & 3.39e-5 \\
    3.375 & 3.60e-8 & 6.875 & 4.91e-5 \\
    3.500 & 3.93e-8 & 7.000 & 7.32e-5 \\
    3.625 & 4.46e-8 & 7.125 & 1.09e-4 \\
    3.750 & 5.14e-8 & 7.250 & 1.76e-4 \\
    3.875 & 5.92e-8 & 7.375 & 2.33e-4 \\
    4.000 & 6.85e-8 & 7.500 & 3.93e-4 \\
    4.125 & 7.82e-8 & 7.625 & 6.13e-4 \\
    4.250 & 9.23e-8 & 7.750 & 9.35e-4 \\
    4.375 & 1.10e-7 & 7.875 & 1.40e-3 \\
    4.500 & 1.27e-7 & 8.000 & 2.35e-3 \\
    4.625 & 1.54e-7 & 8.125 & 3.47e-3 \\
    4.750 & 1.78e-7 & 8.250 & 6.02e-3 \\
    4.875 & 2.26e-7 & 8.375 & 9.91e-3 \\
    5.000 & 2.60e-7 & 8.500 & 1.55e-2 \\
    5.125 & 3.20e-7 & 8.625 & 2.65e-2 \\
    5.250 & 4.17e-7 & 8.750 & 4.54e-2 \\
    5.375 & 5.69e-7 & 8.875 & 7.44e-2 \\
  \bottomrule
\end{tabular}
\end{table}


\newpage
\section{Discussion}
This chapter includes additional plots for the discussion chapter.
% %%%%%%%%%%%%%%%%%%%%%%%%%%%%%%%%%%%%%%%%%%%%%%%%%%%%%%%%%%%%%%%%%%%%%%%%%%%%%
% Differential perf weights for gamma vs logE limits, time dep
% %%%%%%%%%%%%%%%%%%%%%%%%%%%%%%%%%%%%%%%%%%%%%%%%%%%%%%%%%%%%%%%%%%%%%%%%%%%%%
\begin{figure}[H]
  \centering
  \includegraphics{../discussion/plots/discussion_tdep/diff_weights.pdf}
  \caption[Differential performance weights for the largest time window]{
    Weights constructed from the inverse differential performance curve for the largest time window in the time-dependent analysis, as shown in figure~(\ref{fig:tdep_diff_perf}).
    The $\SI{90}{\percent}$ central interval is used to give an impression of the most sensitive region for each power-law hypothesis.
    See section~\ref{chp:tdep_diff_perf} for a derivation of the formulas.
  }
  \label{fig:tdep_diff_weights}
\end{figure}

% %%%%%%%%%%%%%%%%%%%%%%%%%%%%%%%%%%%%%%%%%%%%%%%%%%%%%%%%%%%%%%%%%%%%%%%%%%%%%
% Time dep limits gamma scan normlaisation
% %%%%%%%%%%%%%%%%%%%%%%%%%%%%%%%%%%%%%%%%%%%%%%%%%%%%%%%%%%%%%%%%%%%%%%%%%%%%%
\begin{figure}[H]
  \centering
  \includegraphics{../discussion/plots/discussion_tdep/limits_norm_vs_gammas.pdf}
  \caption[Time-dependent analysis upper limit scan for spectral indices]{
    $\SI{90}{\percent}$ Neyman upper limits for the largest time window in the time-dependent analysis, scanned for a grid of power-laws with different spectral indices.
    The normalisations are shown at two different energies to be able to infer the spectral behaviour.
  }
  \label{fig:tdep_limits_norm_vs_gammas}
\end{figure}

% %%%%%%%%%%%%%%%%%%%%%%%%%%%%%%%%%%%%%%%%%%%%%%%%%%%%%%%%%%%%%%%%%%%%%%%%%%%%%
% Time indep splines for the neyman plane bands
% %%%%%%%%%%%%%%%%%%%%%%%%%%%%%%%%%%%%%%%%%%%%%%%%%%%%%%%%%%%%%%%%%%%%%%%%%%%%%
\begin{figure}[H]
  \centering
  \includegraphics{../discussion/plots/discussion_tindep/neyman_plane_chi2_splines.pdf}
  \caption[Splines for Neyman interval construction in the time-integrated analysis]{
    Smoothing spline fits to the discrete test statistic quantiles for the time-integrated analysis.
    The splines are evaluated to construct more smooth confidence intervals and upper limits than possible from the sampled data alone.
    The parameters are obtained by fitting the modified $\chi^2$ PDF to each slice of the plane.
    The fit parameters are the number of non-zero trials $\eta$, the degrees of freedom d.o.f. and the scale of the distribution.
    The model is defined in formula~(\ref{equ:delta_chi2_pdf}).
  }
  \label{fig:tindep_neyman_plane_chi2_splines}
\end{figure}
\enlargethispage*{5cm}
% %%%%%%%%%%%%%%%%%%%%%%%%%%%%%%%%%%%%%%%%%%%%%%%%%%%%%%%%%%%%%%%%%%%%%%%%%%%%%
% Differential limits for gamma = 2, 3
% %%%%%%%%%%%%%%%%%%%%%%%%%%%%%%%%%%%%%%%%%%%%%%%%%%%%%%%%%%%%%%%%%%%%%%%%%%%%%
\begin{figure}[H]
  \centering
  \includegraphics{../discussion/plots/discussion_tindep/diff_limits.pdf}
  \caption[Time-integrated analysis differential upper limits]{
    Differential limits for the time-integrated analysis for two injection spectra with indices $\gamma_\text{inj}=2$ and $\gamma_\text{inj}=3$.
    The differential limits are not directly comparable to other flux models, because they depend on the binning and the injection index but they can give an overview of the sensitive regions of the analysis.
    The bins are uniformly spaced in logarithmic neutrino energy with a width of one eight of a decade in order to minimize the spectral dependency in each bin.
  }
  \label{fig:tdep_diff_limits}
\end{figure}

% %%%%%%%%%%%%%%%%%%%%%%%%%%%%%%%%%%%%%%%%%%%%%%%%%%%%%%%%%%%%%%%%%%%%%%%%%%%%%
% Differential limit weight PDF for gamma = 2, 3
% %%%%%%%%%%%%%%%%%%%%%%%%%%%%%%%%%%%%%%%%%%%%%%%%%%%%%%%%%%%%%%%%%%%%%%%%%%%%%
\begin{figure}[H]
  \centering
  \includegraphics{../discussion/plots/discussion_tindep/diff_weights_gamma_2.pdf}
  \caption[Differential performance weights for $\gamma_\text{inj}=2$]{
    Weights constructed from the inverse differential performance curve for the injection index $\gamma_\text{inj}=2$ in the time-integrated analysis.
    The $\SI{90}{\percent}$ central interval is used to give an impression of the most sensitive region for each power-law hypothesis.
    See section~\ref{chp:tdep_diff_perf} for a derivation of the formulas.
  }
  \label{fig:tindep_diff_weights_gamma_2}
\end{figure}
\enlargethispage*{5cm}
\begin{figure}[H]
  \centering
  \includegraphics{../discussion/plots/discussion_tindep/diff_weights_gamma_3.pdf}
  \caption[Differential performance weights for $\gamma_\text{inj}=3$]{
    Weights constructed from the inverse differential performance curve for the injection index $\gamma_\text{inj}=3$ in the time-integrated analysis.
    The $\SI{90}{\percent}$ central interval is used to give an impression of the most sensitive region for each power-law hypothesis.
    See section~\ref{chp:tdep_diff_perf} for a derivation of the formulas.
  }
  \label{fig:tindep_diff_weights_gamma_3}
\end{figure}

% %%%%%%%%%%%%%%%%%%%%%%%%%%%%%%%%%%%%%%%%%%%%%%%%%%%%%%%%%%%%%%%%%%%%%%%%%%%%%
% Global limits for gamma inj = 3
% %%%%%%%%%%%%%%%%%%%%%%%%%%%%%%%%%%%%%%%%%%%%%%%%%%%%%%%%%%%%%%%%%%%%%%%%%%%%%
\begin{figure}[H]
  \centering
  \includegraphics{../discussion/plots/discussion_tindep/limits_gamma_logE_gamma_inj_3.pdf}
  \caption[Time-integrated analysis global power-law limits]{
    This analysis' power-law limits compared to limits derived from the diffuse muon neutrino track flux measurement in \cite{Haack:2017dxi} and the all-sky point source search from \cite{Aartsen:2016oji} using seven years of muon neutrino track data.
    The global limits are derived from the differential limits shown in figure~(\ref{fig:tdep_diff_limits}) with an injection index of $\gamma_\text{inj}=3$.
    The double light grey solid line shows the flux limits derived from the diffuse flux measurement scaled to a point source flux.
    The thicker portion of the line shows the valid region given in the original work, the thinner portion the $\SI{90}{\percent}$ central interval obtained from the differential flux limits.
    The double light grey dotted line shows the scaled sensitivity flux from the all-sky point source analysis.
  }
  \label{fig:tdep_limits_gamma_logE_gamma_inj_3}
\end{figure}
