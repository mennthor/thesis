\chapter{Astroparticle physics}

We are able to observe the universe not only in visible light, but in a variety of different wavelengths and other particles.
Typically three main messenger particles are distinguished photons, cosmic rays and neutrinos.
Photon observations range over many decades in wavelength, from radio over microwave, infrared, visible light, ultraviolet, x-ray up the the very highest gamma-ray energies, each used for different purposes and combined to obtain a unified picture of astrophysical processes.
Cosmic rays usually include charged particles as protons, ions or bare nuclei and electrons as well as uncharged neutrons.
While charged particles, especially the heavy particles reach up to the very highest energies observed in the order \SI{100}{\exa\eV}, they usually don't point back to their origin, due to deflection in magnetic field within the intergalactic or interstellar medium.
Neutrons have a half widths time of about \SI{15}{\min} and can't travel far on cosmological scales without decaying and are thus used to probe the closer distances to earth.
Neutrinos on the other hand only interact weakly with normal matter preventing it from getting deflected away from the source position.
This advantage makes it necessary though to build relatively large detectors to be able to register a large enough amount of them.
\needsTODO{More details?}

\section{Astrophysical neutrino production}
\begin{itemize}
  \item Acceleration mechanism of hadronics
  \item Neutrino production from accelerated hadronics
  \item Connection to CR spectrum
\end{itemize}

\section{Possible sources of astrophysical neutrinos}
\begin{itemize}
  \item Unfified AGN model
  \item Supernova remnants
  \item Molecular clouds
  \item Tidal disruptions
  \item GRBs, FRBs -> Transients
  \item What else en vogue currently?
\end{itemize}