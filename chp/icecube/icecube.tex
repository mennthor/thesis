\chapter{Neutrino detection with the IceCube observatory}

\section{Detector}
The low interaction cross section of neutrinos with normal matter makes them not only a suitable astrophysical messenger particle pointing back to production regions, but also makes them hard to detect when they arrive at earth.
Thus there is the need for large detection volumes attacking the low interaction possibilities with larger instrumented volumes.
As the detection principle is based on the photoelectric detection of Cherenkov light in the blue part of the visible spectrum, a detector medium which is optical transparent is needed.
Building a large detection volume from scratch is not feasible so natural resources are used.
Multiple neutrino observatories have been built by instrumenting naturally occurring sites with the needed properties, for example in the Mediterranean sea or in the case of IceCube in the clear Antarctic ice at the South Pole.

IceCube is currently the largest neutrino observatory, instrumenting roughly one cubic kilometer of Antarctic ice.
Instrumentation is realized by an approximately hexagonal grid of 86 strings, each carrying 60 digital optical modules (DOMs).
The DOMs are deployed in depth from \SI{1450}{\m} to \SI{2450}{m} below the ground with a vertical spacing of \SI{17}{\m}.
We further have to distinguish between 80 regular strings with a horizontal spacing of about \SI{125}{\m} in the hexagonal grid and 6 additional DeepCore strings which make up the DeepCore sub-detector.
These strings are more densely packed with a string distance of about \SI{75}{\m} and a vertical spacing of \SI{10}{\m} for 10 DOMs above a naturally occurring dust layer at \SI{2000}{\m} below ground in the ice and the remaining 50 DOMs with a spacing of \SI{7}{\m} below the dust layer, starting at \SI{2100}{\m}.
The denser packing reduces the energy threshold in the sub-array to about \SI{10}{\GeV} suitable to probe atmospheric neutrino oscillation properties.
Additionally there is a also a surface detection array of small ice-filled tanks containing two photomultipliers each, which can be used for cosmic ray studies or as a veto layer for atmospheric particles also traveling down to the IceCube array.
The complete IceCube detector is therefore suitable to detect neutrinos in a wide range of energies, starting at the aforementioned \SI{10}{\GeV} and reaching to the highest energies of multiple \si{\peta\eV} with events lighting up almost the complete detection volume.

Each digital optical module consists of a spherical glass unit withstanding the pressure of the surrounding ice and protecting the electric components inside.
The main detection instrument is a \SI{25}{\cm} diameter photomultiplier facing downwards per DOM.
The signal is digitized and send to the surface data acquisition system, the IceCube Lab, via the string cable.
The main trigger system yields a total trigger rate of approximately \SI{2.8}{\k\Hz}

Construction of the detector began in the Antarctic summer 2004 deploying a single string and ended in 2011 after partial configurations of 1, 9, 22, 40, 59 and 79 strings with the deployment of the last strings to complete the 86 string array.
The sub-arrays deployed each season were already operational so data taking started with the very first configuration.

\section{Detection principle}
Neutrinos come in three lepton flavors, electron-, muon- and tau-neutrino together with their corresponding anti-particle.
As neutrinos only interact weakly with all other standard model particles they cannot be detected directly.
Therefore IceCube detects neutrinos indirectly via the Cherenkov effect.
Neutrinos can undergo neutral current and charged current interaction with matter in the detection volume.
The interactions are the same for all three lepton flavors $l$:
\begin{equation}
  \nu_l + X \rightarrow Z^0 \rightarrow \nu'_l + X'
  \mintertext{and}
  \nu_l + X \rightarrow W^{\pm} \rightarrow l + X'
  \mperiod
\end{equation}

In neutral current interactions neutrinos interact via an intermediate $Z^0$ boson with a nucleus, generally noted as $X$ in the above equation.
The Z boson carries no charge resulting in an unchanged flavor neutrino final state, where only energy is transfered between nucleus and neutrino.
These interactions produce rather spherical, cascade-like signatures in the detector, indistinguishable from which flavor they originated.
Cascade energy

In charged current interactions the exchange particle is the charged $W$ boson, thus changing the initial neutrino state $\nu_l$ to a final lepton state $l$ with the same flavor.
These interaction can produce vastly different event signature in the detector depending on energy.
In general, electron final states produces neutral current like cascades due to their short mean free path and rapid energy loss.
Muon final states can produce track-like signatures and with enough energy travel through the whole detector.
Tau leptons can produce various different signatures depending on the initial energy and particle location.
A unique pattern would be the so-called double bang signature, in which a first cascade is produced in the initial charged current interaction and a second one when the tau lepton decays, depositing energy in the detector medium.

\needsCite{All in Gaisser's new book, chapter 17.}

Despite the different interaction types, the detection principle is the same for all mentioned above.
When energy is deposited in the detector, secondary charged particles are created which, at the high energies considered here, travel through the matter faster than the effective speed of light in the medium.
During their passage trough matter Cherenkov light gets emitted along their path \needsCite{Cherenkov paper / book?}.
The light is emitted in an angle in first order given by the relation
\begin{equation}
  \Theta = \arccos\left(\frac{1}{\beta n}\right)
\end{equation}
with the refractive index $n$ of the traversed medium an $\beta = v/c$ the particle speed in units of the vacuum speed of light $c$.
Typical interactions of the primary charged lepton are ionization, photo nuclear, pair creation and Bremsstrahlung losses where the latter three start to dominate the energy loss for muon energies above some \SI{10}{\tera\eV}.
The Cherenkov light yield from the secondary charged particles originating from the lepton's interaction point is typically higher than the one from the primary lepton itself.

The Cherenkov photons travel through the ice and may be detected in one of the photomultiplier tubes inside the DOMs which are able to detect single photon hits.
The antarctic ice as a natural medium is not perfectly isotropic and has depth and direction dependent absorption and scattering properties which was observed by using artificially injected signal from LEDs build into the DOM cases.
In IceCube we observe a dust layer between a depth of roughly \SI{2000}{\m} and \SI{2100}{m} in which the mean free path is drastically reduced compared to the clear ice in other layers.
Furthermore a directional dependence of the scattering coefficient is present due to effect during ice formation and a vertical tilt of depth layers due to slow glacier movement of the detection volume.
Different ice models have been developed and adapted to account for these different effect as good as possible.
\needsCite{anisotropy, dust layer, icepaper...}

\section{Atmospheric muons and neutrinos}
\needsTODO{Describe the two other main signals that matter here from the physics point of view, air showers}

\begin{itemize}
  \item Air showers from CRs
  \item Muon and munu production in air showers
  \item Generic spectra of atmospheric and astrophysical neutrinos
\end{itemize}