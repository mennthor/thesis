\chapter{Discussion}

Both analyses conducted in this thesis did not yield a significant measurement of a sought-after, lower energy neutrino clustering at the $\num{22}$ proposed high energy starting event positions.
Flux limits based on the non-detection are calculated and presented in this chapter, serving as the final analyses results.

\section*{Time-dependent search}
In the time dependent-search a single parameter fit, the global signal strength, is done in combination with a scan over $\num{21}$, logarithmically increasing time-windows from two seconds to five days.
After trial-correction, the measurement in the largest time window yielded a non-significant result of $\SIsigma{1.03}$.
Physics results from non-significant measurements are obtained by the creation of Neyman upper limits on the measured parameter space.

First, upper limits are calculated for each time window separately.
This is done to show the flux constraints if each window would not have been scanned, but a priori fixed.
To compute the limits for a confidence level $\alpha$, the amount of signal needs to be found, so that the test statistic with injected signal events has its $1-\alpha$ percentile at the measured test statistic value in the corresponding time window.
Here, upper limits at the usual $\SI{90}{\percent}$ confidence level are chosen.
The resulting upper limits for each time window can be seen in figure~(\ref{fig:time_dep_pre_trial_limits}).
For each time window with a test statistic of zero, the sensitivity is automatically the upper limit.
For the largest time window, the upper limit is higher because of the non-zero significance in that region.
The upper limit is calculated for the fixed power-law with spectral index $-2$.

To obtain limits for a wider range of spectra and not only the one that is explicitly tested against, limits are calculated for several different spectral indices for the assumed sources and for all time-windows.
Tested is always against the largest time-window with the highest significance.
The resulting upper limits can be seen in figure~(\ref{fig:time_dep_gamma_scan_limits}).

The calculated limits are not directly comparable to a specific model, as models that predict a time-dependent flux or a per-burst emission from a possible source of high energy starting events do not exist.
Time-dependent and burst models are primarily built for Gamma Ray Bursts, which are tested in a dedicated time-dependent analysis similar to this work \cite{Aartsen:2014aqy,Abbasi:2012zw}.
However, it is possible to compare the limits against the current diffuse astrophysical muon neutrino flux measurement made in IceCube \cite{Haack:2017dxi}.
The used flux model is an unbroken power law and the best fit is given as
\begin{equation}
  \label{equ:diffuse_flux}
  \phi^{\nu_\mu+\bar{\nu}_\mu}
  = \left(1.01_{-0.23}^{+0.26}\right) \cdot
    \left(\frac{E}{\SI{100}{\TeV}}\right)^{−2.19\pm0.10} \cdot
    \SI{e-18}{\per\GeV\per\cm\squared\per\second\per\steradian}
  \mperiod
\end{equation}

This can be calculated in a fluence upper limit that can be used to compare this analysis upper limit against.
Under the assumption, that the whole diffuse flux from~(\ref{equ:diffuse_flux}) is accounted for by burst emission originating from the total $\num{82}$ high energy starting event positions in the 6 years of IceCube data.
This assumption leads to a burst rate of
\begin{equation}
  R_\text{HESE}
  \approx \frac{82}{4\pi\cdot6\cdot365\cdot\num{86400}}
    \si[per-mode=fraction]{\per\second\per\steradian}
  \approx \SI{3,44e-8}{\per\second\per\steradian}
\end{equation}
for a single direction per second.
The diffuse flux normalization at $\SI{100}{\TeV}$
\begin{equation}
  \phi_0^{\nu_\mu+\bar{\nu}_\mu}
  = \SI{1.01e-18}{\per\GeV\per\cm\squared\per\second\per\steradian}
\end{equation}
can then be scaled to a per-burst fluence with
\begin{equation}
  \Phi_\text{HESE}^\text{diff}
  = \frac{\Delta T \cdot \phi_0^{\nu_\mu+\bar{\nu}_\mu} \cdot
          \SI{4\pi}{\steradian}}
         {\Delta T \cdot R_\text{HESE} \cdot \SI{4\pi}{\steradian}}
  = \frac{\phi_0^{\nu_\mu+\bar{\nu}_\mu}}{R_\text{HESE}}
  \mperiod
\end{equation}
Note, that the dependence on the burst time window $\Delta T$ and the factor $4\pi$ for the isotropic integration over the whole sky is cancelled, because the burst rate scales accordingly.
This leads to a per-burst fluence of
\begin{equation}
    \Phi_\text{HESE}^\text{diff} \approx \SI{0.29}{\per\GeV\per\cm\squared}
\end{equation}
at at a normalization energy of $\SI{100}{\TeV}$.
Any fluence higher than $\Phi_\text{HESE}^\text{diff}$ would mean that more than the total observed, diffuse astrophysical muon neutrino flux would originate from a possible per-burst emission at the HESE event positions.
As seen in figure~(\ref{fig:sensitivity_with_diffuse}), the limits set by this analysis are well below the diffuse limits, indicating a smaller contribution to the diffuse flux, if any at all.

\section*{Time-integrated search}
The search for a steady flux emission originating from the $\num{22}$ high energy starting event positions also yields no significant result.
A slight over-fluctuation of $\SIsigma{0.8}$ over the expected background could be measured in six years of muon track data.


\begin{itemize}
  \item Analysis results, plain numbers, orient at FRB or GRB for example.
  Both for time-independent and dependent stacking
  \item present the limit scan pre-trial and post-trial, try to add a somewhat meaningful comparison curve to the plots, eg. diffuse HESE or diffuse Aachen.
  \item Include LLH scan and interval for ns parameter for the unblinded data.
\end{itemize}

\begin{figure}[htbp]
  \centering
  \includegraphics{plots/bf_llh_scan.pdf}
  % \subimport*{plots/}{bf_llh_scan_rasterized.pdf}
  \caption[2D LLH scan of $n_S$ and $\gamma$ for the time integrated search.]{
    2D Likelihood scan around the experimental best-fit position in spectral index $\gamma$ and expected signal events $n_S$, with the experimental result marked by the white dot.
    The contours are taken according to Wilks' theorem from a $\chi^2$ distribution with $\num{2}$ degrees of freedom.
    $\SI{39}{\percent}$ of probability are enclosed by the $\SI{1}{\sigma}$ contour.
    The dashed line gives the marginal scan line for fixing the spectral index and only fitting $n_S$, the dotted line vice versa for a fixed $n_S$.}
\end{figure}
