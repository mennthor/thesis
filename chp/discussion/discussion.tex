\chapter{Discussion}

\begin{itemize}
  \item Analysis results, plain numbers, orient at FRB or GRB for example.
  Both for time independent and dependent stacking
  \item present the limit scan pre-trial and post-trial, try to add a somewhat meaningful comparison curve to the plots, eg. diffuse HESE or diffuse Aachen.
  \item Include LLH scan and interval for ns parameter for the unblinded data.
  % Physics part
  \item Compare emission scenarios to (somewhat) likely scenarios from the intro section, sort in to physics context or compare to other analysis???
  \item Compare both results in light of the TXS result, trend to larger time windows
\end{itemize}

% \begin{figure}[htbp]
%   \centering
%   \includegraphics{plots/bf_llh_scan_rasterized}
%   % \subimport*{plots/}{bf_llh_scan_rasterized.pdf}
%   \caption{
%     2D Likelihood scan around the experimental best fit position in spectral index $\gamma$ and expected signal events $n_S$, with the experimental result marked by the white dot.
%     The contours are taken according to Wilks' theorem from a $\chi^2$ distribution with $\num{2}$ degrees of freedom.
%     $\SI{39}{\percent}$ of probability are enclosed by the $\SI{1}{\sigma}$ contour.
%     The dashed line gives the marginal scan line for fixing the spectral index and only fitting $n_S$. The dotted line vice versa with fixing $n_S$.}
% \end{figure}

