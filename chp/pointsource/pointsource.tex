\chapter{Point source searches with IceCube neutrinos}
In this chapter the unbinned Likelihood search methods used for multiple IceCube point source searches are derived.
To identify an astrophysical neutrino signal from a specific source location on the sky an excess of events from that direction needs to be identified.
In general it cannot be distinguished between atmospheric and the sought after extraterrestrial neutrinos on a per event basis.
Instead it is possible to search from a sample based point-of-view by measuring deviations for an ensemble of events with respect to a known background expectation.
A simple method could be to define a fixed search region around a direction from which signal is expected, count the number of measured events and compare them to number of events from a background expectation in the region.
\CITE{Li Ma Paper.}
If the measured number of events is significantly higher than the expected number of background events that may be a hint for a signal from that direction.

Here the followed approach is similar but used in a more advanced form, incorporating multiple event informations to increase the detection sensitivity.
Also the usage of pre-defined search regions, often also called a binned approach, is replaced with an unbinned version on a per event basis.
This has the advantage of avoiding hard search region boundaries which can drop the sensitivity of the approach if e.g. an unknown source lies directly at the border of such a region.
Starting from a general unbinned, extended Likelihood approach, the special cases used in this analysis to carry out a time dependent stacking search for point-like sources are derived.
Particular cases handling multiple years of data from different detector configurations and multiple sources for the so called stacking case are derived from the basis form.
In the following a single event is noted with index $i$, a source with index $k$ and a data sample with index $j$.

\FIXME{Describe statistics framework, frequentist p-values, a-priori sensitivity, chi2 method or later in the actual results section?}

\section{Extended unbinned Likelihood}
The extended Likelihood \CITE{Barlow book} and the corresponding logarithmic extended Likelihood function is defined as
\begin{equation}
  \mathcal{L}(\lambda) = \frac{\lambda^N e^{-\lambda}}{N!}\prod_{i=1}^N P_i
  \quad\Rightarrow\quad
    \ln\mathcal{L}(\lambda) = -\lambda+\sum_{i=1}^N \ln(\lambda P_i)
  \mcomma
\end{equation}
where the constant term $\ln(N!)$ is dropped in the logarithmic version.
Here $\lambda$ is the expected number of events and $N$ the number of measured events following a Poisson counting distribution.
The per event model distribution $P_i$, normalized to integral $1$ over the defined parameter space, describes the Likelihood of each event under the assumed model and how likely it contributes to the expectation.
The use of the Poisson term is justified by a re-normalization of the per event distributions to include the total number of measured events which is not fixed for multiple experiments of the same kind but may fluctuate around an unknown expectation value.

The tested hypotheses are encoded in the description of the models $P$.
To obtain a fairly general expression to derive the point source special cases from, the expectation model can be split in multiple classes by splitting the expectation and the models accordingly
\begin{equation}
  \lambda = \sum_{k=1}^{N_\text{classes}} \lambda_k \geq 0
  \mintertext{and}
  P_i = \frac{1}{\sum_{k=1}^{N_\text{classes}} \lambda_k}\cdot
         \sum_{k=1}^{N_\text{classes}} \lambda_k P_{i,k}
  \mperiod
\end{equation}
The single $\lambda_k$ can be negative but their sum must not, because it is still a Poisson expectation parameter.
Additionally the new split model is normalized over all classes to arrive at the form
\begin{equation}
  \ln\mathcal{L}(\{\lambda_k\})
  = -\sum_{k=1}^{N_\text{classes}} \lambda_k +
    \sum_{i=1}^N \ln\left(\sum_{k=1}^{N_\text{classes}}
      \lambda_k P_{i,k} \right)
  \mperiod
\end{equation}

To specialize more, it is usually desired to test a signal hypothesis against a background one, for $N_\text{srcs}$ sources in general, for each event $i$.
The above expression is thus expanded to include $N_\text{srcs}$ signal and $N_\text{srcs}$ background parameters and the corresponding distributions $S_{i,k}$ and $B_{i,k}$:
\begin{equation}
  \ln\mathcal{L}(\{\lambda_{k,S}\}, \{\lambda_{k,B}\})
  = -\sum_{k=1}^{N_\text{srcs}}\left(\lambda_{k,S}+\lambda_{k,B}\right) +
    \sum_{i=1}^N \ln\left(\sum_{k=1}^{N_\text{srcs}}\left(
      \lambda_{k,S} S_{i,k}+\lambda_{k,B} B_{i,k}\right)\right)
  \mcomma
\end{equation}
and from the Poisson condition there is still the constrain
\begin{equation}
  \sum_{k=1}^{N_\text{srcs}}\left(\lambda_{k,S}+\lambda_{k,B}\right) \geq 0
  \mperiod
\end{equation}

For testing the significance of a potential signal contribution in the measured data, a Likelihood ratio test is used.
The null hypotheses $H_0$, which is that only background is expected to be measured, is constructed by using only a portion $\Theta_0$ of the allowed parameter space, here by setting all signal expectations to zero
\begin{equation}
  \ln\mathcal{L}_0(\{\lambda_{k,S}=0\}, \{\lambda_{k,B}\})
  = -\sum_{k=1}^{N_\text{srcs}}\left(\lambda_{k,B}\right) +
    \sum_{i=1}^N \ln\left(\sum_{k=1}^{N_\text{srcs}}\left(
      \lambda_{k,B} B_{i,k}\right)\right)
  \mperiod
\end{equation}
The alternative hypothesis $H_1$ is constructed using the full Likelihood parameter space $\Theta$:
\begin{equation}
  \ln\mathcal{L}_1(\{\lambda_{k,S}\}, \{\lambda_{k,B}\})
  = -\sum_{k=1}^{N_\text{srcs}}\left(\lambda_{k,S}+
                                     \lambda_{k,B}\right) +
    \sum_{i=1}^N \ln\left(\sum_{k=1}^{N_\text{srcs}}\left(
      \lambda_{k,S} S_{i,k}+\lambda_{k,B} B_{i,k}\right)\right)
  \mperiod
\end{equation}

The Likelihood ratio test statistic $\Lambda$ for testing the null hypothesis $H_0$ against the alternative $H_1$ is defined as \CITE{Casella Berger Book}
\begin{equation}
  \ln\Lambda = \ln\left(\frac{\sup_{\theta \in \Theta_0} \mathcal{L}(\theta)}
                          {\sup_{\theta \in \Theta} \mathcal{L}(\theta)}\right)
  = \ln\left(\sup_{\theta \in \Theta_0} \mathcal{L}(\theta)\right) -
    \ln\left(\sup_{\theta \in \Theta} \mathcal{L}(\theta)\right)
  \mperiod
\end{equation}

Here the parameters $\hat{\lambda}_{k,S/B}$ were introduced, which mean the parameters $\lambda_{k,S/B}$ that maximize the Likelihood $\mathcal{L}_1$ under the complete parameter space and $\hat{\lambda}_{k,B}^{(0)}$ maximizing $\mathcal{L}_0$.
This leads to the test statistic
\begin{equation}
  \begin{aligned}
    -2\ln\Lambda
    &= 2\ln(\mathcal{L}_1(\{\hat{\lambda}_{k,S/B}\})) -
       2\ln(\mathcal{L}_0(\{\hat{\lambda}^{(0)}_{k,B}\})) \\
    &= -2\left(\sum_{k=1}^{N_\text{srcs}}\hat{\lambda}_{k,S} +
                                         \hat{\lambda}_{k,B} -
                                         \hat{\lambda}_{k,B}^{(0)}\right) +
      2\sum_{i=1}^N \ln\left(
        \frac{\sum_{k=1}^{N_\text{srcs}}\left(
            \hat{\lambda}_{k,S} S_{i,k}+\hat{\lambda}_{k,B} B_{i,k}\right)}
            {\sum_{k=1}^{N_\text{srcs}}\left(
              \hat{\lambda}_{k,B}^{(0)} B_{i,k}
            \right)}
          \right)
    \mcomma
  \end{aligned}
\end{equation}
which has been decorated by the factor $-2$ to be compatible to Wilks' theorem \CITE{Wilk paper and/or statistic book}.
As seen above, all best fit parameters from both hypotheses have to be distinguished in general, differentiating $\hat{\lambda}_{k,B}^{(0)}$ from the null hypothesis which are not the same as $\hat{\lambda}_{k,B}$ from the composite hypothesis.

In the following sections specific model choices for the signal and background distributions and approximating assumptions are shown, to transform the above expression to commonly known forms also used for the time dependent search in this thesis.


% %%%%%%%%%%%%%%%%%%%%%%%%%%%%%%%%%%%%%%%%%%%%%%%%%%%%%%%%%%%%%%%%%%%%%%%%%%%%%
% %% General method to get the PDFs, not the used implementation
% %%%%%%%%%%%%%%%%%%%%%%%%%%%%%%%%%%%%%%%%%%%%%%%%%%%%%%%%%%%%%%%%%%%%%%%%%%%%%
\section{Per event distributions}
The introduced per event distributions $S_i, B_i$ are the functions that actually define the tested hypothesis of the analysis.
Depending on their structure they can describe a single point source search, a stacking search for searching at various source positions at once or a template search, where a whole spatial region is tested for neutrino emission over the expectation.
The used PDFs are usually similar for all analysis types and the conventions applied for most point source searches in IceCube are followed here \CITE{Braun paper. Maybe both for time dep. / indep. searches.}.

The per-event distributions introduce the main separation power between signal and background hypotheses in combination with the mixing portions $\lambda_{i,S/B}$ by introducing a-priori knowledge in defining signal- and background-like regions in the tested parameter space.
The better these distributions are able to separate signal and background regions the more sensitive the analysis becomes.
A common approach with known good separation power is to combine contributions from spatial clustering and energy information, where the first one is necessary for the tested hypotheses and the latter providing additional information under certain assumptions of signal flux shapes.
For time dependent analysis an extra time dependent part is introduced.

For the time dependent analysis implemented here, the signal and background contributions can be written as independent products of a spatial, an energy and a time dependent part as
\begin{align}
  S_{i,k}
    &= S(\vec{x}_i, \vec{x}_{\mathrm{src},k}, E_i | \gamma)
     = S^S(\vec{x}_i, \vec{x}_{\mathrm{src},k}) \cdot
       S^E(E_i, \delta_i | \gamma) \cdot
       S^T(t_i, t_k) \\
  \intertext{and}
  B_{i,k}
    &= B(\delta_i, E_i | \phi_\mathrm{BG})
     = B^S(\delta_i) \cdot
       B^E(E_i, \delta_i | \phi_\mathrm{BG}) \cdot
       B^T(t_i, t_k)
\end{align}
where $\gamma$ is the shape parameter of an signal flux usually assumed to be a power law $\propto E^{-\gamma}$ and $\phi_\mathrm{BG}$ stands for a flux model of the atmospheric neutrino flux describing the background flux dependency.
The event times $t_i$ and source times $t_k$ define the time dependent emission model.

\subsection{Spatial distribution}
The most important part in the search for neutrino point sources is the spatial clustering of events around a point on the sky.
For data samples used in point source searches there is an reconstructed estimate of the per event uncertainty available.
This estimator is built from the positional reconstruction Likelihood fit and is constructed assuming a symmetric, two dimensional Gaussian distribution describing the reconstruction uncertainty \CITE{Paraboloid}.
Alternatively the same value can be transformed and used for a symmetric Kent distribution \CITE{Kent paper} for the per event spatial distribution instead.
The van-Mises distribution is the pendant of a symmetric two dimensional Gaussian but correctly normalized on the sphere and is useful for larger uncertainties for example when investigating cascade-like events.
For tracks the good angular resolution of about \SI{1}{\degree} justifies the use of the simpler and more familiar Gaussian distribution as both distributions become virtually indistinguishable for small uncertainties.
\FIXME{Appendix: Show or ref to plot of sigma vs kappa}
In general the joint probability for an event $i$ to spatially originate from a source $k$ can be obtained by a convolution of the two separate spatial PDFs.
In a search for point sources at known positions \CITE{coenders, allsky scan}, the distribution of a source position is represented by a delta distribution.
The resulting spatial signal distribution describing the probability of an event being spatially correlated to a given source can thus be expressed as a convolution of the Gaussian per event part and a delta distribution for the fixed source position
\begin{equation}
  \begin{aligned}
    S^S(\vec{x}_i, \vec{x}_{\mathrm{src},k}) &=
      \int_\Omega \frac{1}{2\pi \sigma_i^2}
      \exp\left(-\frac{|\vec{x}-\vec{x}_i|^2}{2\sigma_i^2}\right) \cdot
      \delta(\vec{x}_{\mathrm{src},k} - \vec{x}) \d{\vec{x}} \\
      &= \frac{1}{2\pi \sigma_i^2}
         \exp\left(-\frac{|\vec{x}_{\mathrm{src},k}-
                          \vec{x}_i|^2}{2\sigma_i^2}\right)
      \mperiod
  \end{aligned}
\end{equation}
Assuming extended sources with a Gaussian density profile would follow the same scheme and an analytic solution for the convolution of two Gaussian distributions exist, resulting in a substitution of $\sigma_i \rightarrow \sqrt{\sigma_i^2 + \sigma_{\text{src},k}^2}$ \CITE{Gaussian convolution}.

The spatial background distribution is constructed similarly to the signal case above.
Because IceCube is a nearly right-ascension symmetric detector at the south pole, the distribution is assumed to be declination dependent only.
This may break down for time scales so small, that the earth's rotation doesn't smear out the slightly asymmetric distribution in azimuth angle, which can then be used as a drop-in replacement, but holds better the larger the regarded time windows get.
This dependency is written in a general form here and can for example be constructed by constructing histograms of experimental data, which is described later in more detail.
The background spatial distribution can then be written as
\begin{equation}
  B^S(\delta_i) = \frac{1}{2\pi} \cdot P(\sin(\delta_i))
  \mcomma
\end{equation}
where the first factor is the uniform distribution in right-ascension and the latter indicating, the often alternatively used, dependence on $\sin(\delta)$.
Note that the background distribution is only depending on the event position because background should have no correlation with any source by definition.

\subsection{Energy distribution}
In addition to the spatial clustering the event's energy can also provide a powerful separation argument and lead to a large improvement in sensitivity \CITE{Braun Paper}.
As the energy density of atmospheric neutrinos can approximately be described by a power law $\phi_\mathrm{BG}(E) \propto E^{-3.7}$ and astrophysical signal by a harder spectrum around $\phi(E)_S \propto E^{-2.2}$ \CITE{latest ICRC Aachen diffuse? Or generic -2?}, higher energy events are more likely to originate from a extraterrestrial source rather than having been created in the atmosphere.
The energy dependence can only be taken into account with energy estimators of the the true neutrino energy $E_\nu$.
Formally that can be written as a integration using the law of total probabilities of the distribution of the energy proxy of the event with the probability of obtaining a true neutrino energy under the current flux hypothesis
\begin{equation}
  S^E(E_i, \delta_i|\gamma) =
    \int_0^\infty P(E_i,\delta_i|E_\nu)\cdot P(E_\nu|\gamma) \d{E_\nu}
    \mcomma
\end{equation}
where $\gamma$ is the shape parameter of an assumed signal power law flux.

For background the same reasoning applies and the flux model is substituted for one describing the atmospheric neutrino background instead
\begin{equation}
  B^E(E_i, \delta_i|\phi_\mathrm{BG}) =
    \int_0^\infty P(E_i,\delta_i|E_\nu)\cdot P(E_\nu|\phi_\mathrm{BG}) \d{E_\nu}
    \mperiod
\end{equation}

In practice the integrals may be obtained using histograms in declination and an energy estimator from simulations for signal and from simulations or measured data for background.

\subsection{Time dependency}
To test for time dependent emission models, also time dependent PDFs $S_{i,k}^T(t_i, t_k)$ and $B_{i,k}^T(t_i, t_k)$ which depend on each events time and it's occurrence relative to the sources' temporal occurrence can be included.
The assumption is then, that each source only emits neutrinos as given by $S_{i,k}^T(t_i, t_k)$.
The background can in general also be time dependent to account e.g. for seasonal variations in the detector rate.


% %%%%%%%%%%%%%%%%%%%%%%%%%%%%%%%%%%%%%%%%%%%%%%%%%%%%%%%%%%%%%%%%%%%%%%%%%%%%%
% %% Special case GRB LLH
% %%%%%%%%%%%%%%%%%%%%%%%%%%%%%%%%%%%%%%%%%%%%%%%%%%%%%%%%%%%%%%%%%%%%%%%%%%%%%
\section{Time dependent Likelihood}
To test for time dependent emission in this analysis, the general extended Likelihood is altered to a form similar to what is usually called \emph{Gamma Ray Burst Likelihood}\footnote{Named after the original purpose of searching for emission from a Gamma Ray Burst catalogue. The concept can be applied to other sources testing a similar hypothesis of emission within a defined time window.} \CITE{Mikes thesis or GRB paper or FRB paper}.
This includes explicit assumptions about the temporal source emission PDFs and simplifications of the general Likelihood formula introduced before.
The simplifications need to be applied mainly because when choosing small time windows the analysis deals with a very low amount of leftover events.
Therefore the number of free parameters needs to be reduced as much as possible without introducing too much a-priori assumptions to stick to a rather general search method, because the source types themselves are unknown.

Here, a rather general assumption about the source emission is used by choosing rectangle functions for the time windows, having only a non-zero contribution within pre-defined source time windows
\begin{equation}
  S_{i,k}^T = B_{i,k}^T = T_k(t_i) \coloneqq
    \rect\left(\frac{t_i - \frac{t_k^1-t_k^0}{2}}
                              {t_k^1-t_k^0}\right)
  \mcomma
\end{equation}
which effectively cuts out a subset of events around the sources time stamps $t_k$ and the corresponding time interval around each source $[t_k^0, t_k^1]$.
The most simple case is then to have each source in its own, non-overlapping time window so that each source has a unique set of events belonging to it, which is the case in this analysis.
Note that no separation power stems from these time PDFs, they are merely used to reduce the background rate under the assumption of a temporally concentrated emission.

One important simplification of the general Likelihood that is applied, is that the background expectations are not fitted, but rather fixed from the integrated off-time data rate over the range of the background time PDFs.
This decreases the number of parameters to fit for, because it unifies and fixes the background estimators to
\begin{equation}
  \hat{\lambda}_{k,B} = \hat{\lambda}_{k,B}^{(0)} = \Braket{\lambda_{k,B}}
  \mperiod
\end{equation}
The test statistic then turns to the form
\begin{equation}
  -2\ln\Lambda
  = -2\sum_{k=1}^{N_\text{srcs}}\hat{\lambda}_{k,S} +
      2\sum_{i=1}^N \ln\left(
        \frac{\sum_{k=1}^{N_\text{srcs}}\hat{\lambda}_{k,S} S_{i,k}}
             {\sum_{k=1}^{N_\text{srcs}}\Braket{\lambda_{k,B}} B_{i,k}}
        + 1
      \right)
  \mperiod
\end{equation}

The last Likelihood simplification performed, in necessity to that a large number of free parameters cannot be fitted to very few events in a low background analysis, is to fix the relative signal expectations of each source a-priori and only fit for the total expectation
\begin{equation}
  \lambda_{k,S} = n_S \cdot w_k
\end{equation}
with a free global signal strength parameter $n_S$ and weights $w_k$ normalized so that
\begin{equation}
  \sum_{k=1}^{N_\text{srcs}} w_k = 1
  \mperiod
\end{equation}
The test statistic can then be expressed by
\begin{equation}
  -2\ln\Lambda
  = -2\hat{n}_S +
      2\sum_{i=1}^N \ln\left(
        \frac{\hat{n}_S \sum_{k=1}^{N_\text{srcs}} w_k S_{i,k}}
             {\sum_{k=1}^{N_\text{srcs}}\Braket{\lambda_{k,B}} B_{i,k}}
        + 1
      \right)
  \mperiod
\end{equation}

The a-priori fixed weights resemble the expectation from a each single source at the detector.
They can therefore depend on the explicitly chosen source emission model and the detection efficiency depending on the source location in the detector.
Note that the a-priori chosen weights should match the true, but usually unknown, emission scenario as much as possible to obtain a good analysis sensitivity.
If the true scenario strongly differs from the assumed weights, very little can be obtained from the analysis \CITE{Rameez thesis}.


\subsection{Single sample stacking case}
Shown here is, that the assumption of independent time windows for the sources simplifies the test statistic further.
The expression can be rearranged to an explicit sum of logarithms, that better resembles the uniqueness of each source in it's time window.

Starting from the test statistic from above
\begin{equation}
  -2\ln\Lambda
  = -2\hat{n}_S +
      2\sum_{i=1}^N \ln\left(
        \frac{\hat{n}_S \sum_{k=1}^{N_\text{srcs}} w_k S_{i,k}}
             {\sum_{k=1}^{N_\text{srcs}}\Braket{\lambda_{k,B}} B_{i,k}}
        + 1
      \right)
  \mcomma
\end{equation}
and from the definition of the unique time windows, it can be seen that each event $i$ can only contribute to a single source $k$.
The event belongs to the source in which time window it falls or to no source at all, where the time windows are defined as above
\begin{equation}
  T_k(t_i) \coloneqq \rect \left(
    \frac{t_i - \frac{t_k^1-t_k^0}{2}} {t_k^1-t_k^0}
  \right)
  \mperiod
\end{equation}
The stacking sum then turns to
\begin{align}
  -2\ln\Lambda
  &= -2\hat{n}_S +
      2\sum_{i=1}^N \ln\left(
        \frac{\hat{n}_S \sum_{k=1}^{N_\text{srcs}} w_k S_{i,k}
              \delta_{\{i,k|T_k(t_i)\neq 0\}}}
             {\sum_{k=1}^{N_\text{srcs}}\Braket{\lambda_{k,B}} B_{i,k}
              \delta_{\{i,k|T_k(t_i)\neq 0\}}}
        + 1
      \right) \\
  &= -2\hat{n}_S +
      2\sum_{i=1}^N \ln\left(
        \frac{\hat{n}_S \left[0+\dots+0+ w_{k*} S_{i,k*} +0+\dots+0\right]}
             {\left[
              0+\dots+0+ \Braket{\lambda_{k*,B}} B_{i,k*} +0+\dots+0
              \right]}
        + 1
      \right) \\
  &= -2\hat{n}_S +
      2\sum_{i=1}^N \sum_{k=1}^{N_\text{srcs}} \ln\left(
        \frac{\hat{n}_S w_k S_{i,k}}{\Braket{\lambda_{k,B}} B_{i,k}}
        + 1
      \right) \mcomma
\end{align}
where $k*$ mean the $k$ that fulfils the condition $T_k(t_i)\neq 0$ and in the last step it is used that
\begin{equation}
   \ln\left(
      \frac{\hat{n}_S w_{k\neq k*} S_{i,k\neq k*}}{\Braket{\lambda_{k\neq k*,B}} B_{i,k\neq k*}}
      + 1 \right) = \ln(0 + 1) = 0
      \mperiod
\end{equation}

\subsection{Multiple samples stacking case}
To add more sources in the unique time window scenario, data at the times at which the source events occurred needs to be tested.
Because after each year of taking IceCube data there may be a change in the data taking procedure, these changes must be included in the expectations for the samples used in the single Likelihood test.
These differences can be considered in an additional weighting scheme.
The reasoning is quite similar to the stacking case for the single sample Likelihood from before.

To add another sample, the individual Likelihoods can be summed up, because the tested datasets are independent, which yields the combined test statistic
\begin{align}
  -2\ln\Lambda
  &= \sum_{j=1}^{N_\text{sam}} -2\ln\Lambda_j(\hat{n}_{S,j}) \\
  &= \sum_{j=1}^{N_\text{sam}} \left[
        -2\hat{n}_{S,j} +
        2\sum_{i=1}^N \sum_{k=1}^{N_\text{srcs}} \ln\left(
          \frac{\hat{n}_{S,j} w_k S_{i,k}}{\Braket{\lambda_{k,B}} B_{i,k}}
          + 1
        \right)
      \right]
  \mcomma
\end{align}
where individual free $n_{S,j}$ signal parameters are introduced and the weights $w_k$ are normalized per sample.
Again, a-priori information about the expected number of signal events originating from each sample can be used and a global free signal strength parameter $n_S$ is used with
\begin{equation}
  n_{S,j} = w_j n_S
  \mperiod
\end{equation}
To calculate the a-priori weights $w_j$ the law of total probability is applied \CITE{What to cite here? Casella Berger again or Blobel?}
\begin{equation}
  w_j = P(j) = \sum_{k=1}^{N_\text{srcs}} P(j|k)\cdot P(k)
  \mcomma
\end{equation}
because only the conditional probability contribution from each source $k$ per sample $j$ is known.
That means, the needed probability $P(j)$ of getting a signal contribution $n_S w_j$ from sample $j$ splits into $P(j|k)$ and $P(k)$.
$P(j|k)$ is the probability of getting signal from source $k$ within sample $j$, normalized over all samples
\begin{equation}
  \sum_{j=1}^{N_\text{sam}} P(j|k) = 1
  \mperiod
\end{equation}
Additionally, $P(k)$ is the probability of getting signal from source $k$ at all within any sample, separately normalized over all sources separately
\begin{equation}
  \sum_{k=1}^{N_\text{srcs}} P(k) = 1
  \mperiod
\end{equation}
These relations can also be written in a concise matrix notation
\begin{equation}
  \begin{pmatrix} w_1 \\ \vdots \\ w_{N_\text{sam}} \end{pmatrix} =
    \begin{pmatrix}
      P(j=0|k=0) & \dots & P(j=0|k=N_\text{srcs}) \\
      \vdots & \ddots & \vdots \\
      P(j=N_\text{sam}|k=0) & \dots & P(j=N_\text{sam}|k=N_\text{srcs})
    \end{pmatrix} \cdot
    \begin{pmatrix}
      P(k=0) \\ \vdots \\ P(k=N_\text{srcs})
    \end{pmatrix}
  \mperiod
\end{equation}
The unnormalized signal expectation values can be obtained in each sample by calculating the expected number of events from a signal simulation which usually differs for each detector configuration and sample selection criteria.
These values can then be used to normalize the matrix per column and construct the $P(K)$ vector by summing over each columns per source $k$.

The most complex weighting case in this scenario would be multiple sources with time PDFs overlapping in their emission region and also leaking into another data sample.
The formalism above still applies, and the sample splitting weighs can be obtained by integrating the time emission PDFs per sample to obtain the relative emission strength for the source portions lying in each sample.
These are then multiplied with the usual declination dependent weights per sample to form the $P(j|k)$ entries of the above matrix.
The weighting within each sample wouldn't be affected and is still valid as explained in the previous section.

For the special case treated here, with each source having it's unique time window and also falling exclusively in a single sample, each column has only a single entry which is $1$ after the trivial normalization. \TODO{Maybe put complete and explicit matrix and weights for the 22 HESE sources in the appendix.}
The probabilities $P(k)$ can be obtained by using the un-normalized $n_S$ splitting weights for each sample $\tilde{w}_k$ and re-normalize them over all sources in all samples
\begin{equation}
  P(k) = \frac{\tilde{w}_k}{\sum_{m=1}^{N_\text{srcs}} \tilde{w}_m}
  \mperiod
\end{equation}
Because of the special matrix properties used here, the explicit weights $w_j$ then turn out to be global re-normalizations of the un-normalized single sample weights $\tilde{w}_k$ with
\begin{align}
  w_j
    &= \sum_{k=1}^{N_\text{srcs}} P(j|k)\cdot P(k) \\
    &= \sum_{k=1}^{N_\text{srcs}}
      \delta_{\{k,j|T_j(t_k)\neq 0\}} \cdot
      \frac{\tilde{w}_k}{\sum_{m=1}^{N_\text{srcs}} \tilde{w}_m}
  \mcomma
\end{align}
where $T_j$ is a unique rectangle function for each sample, equally used as the rectangle function utilized to describe the unique time windows per source in each sample.

Now the numerator turns out to be exactly the per sample normalization of the per sample splitting weights, which is the sum of all weights for the subset of all sources that actually are in the sample.
The full multi-sample test statistic then reads
\begin{equation}
  -2\ln\Lambda
  = -2\hat{n}_S +
      2\sum_{j=1}^{N_\text{sam}} \sum_{i=1}^N \sum_{k=1}^{N_\text{srcs}}
      \ln\left(
        \frac{\hat{n}_{S}\frac{\tilde{w}_k}{\sum_{m=1}^{N_\text{srcs}}
              \tilde{w}_m} S_{i,k}}
             {\Braket{\lambda_{k,B}} B_{i,k}}
        + 1
      \right)
  \mcomma
\end{equation}
where $\tilde{w}_k$ are the un-normalized weights per source with respect to the expected signal in their corresponding sample and are normalized over all source expectations in all samples regarded.
This expression nicely demonstrates the circumstances here, namely that each source is independent of each other source and lies completely in a single data sample, so the whole underlying Likelihood fully factorizes in events, sources and samples.

% \subsection{Summary}
% We have explored two ways to construct a stacking GRB LLH with multiple samples:
% \begin{enumerate}
%   \item Construct the single sample LLHs normalizing all weights as shown in the first chapter in combination with the multiply LLH in eq.~(\ref{equ:multi_TS}).
%   Then we need to introduce sample weights $w_j$ for $\hat{n}_s$, so that
%   \begin{equation}
%     w_j = \sum_{k\in j} w^\text{D}_k w^\text{T}_k
%           \frac{1}{\sum_{m=1}^{N_\text{srcs}} w^\text{D}_m w^\text{T}_m}
%   \end{equation}
%   These weights re-normalize the per sample weights so they are regarding different efficiencies across samples.

%   \item Alternatively we can also use form eq.~(\ref{equ:single_stack}), which is exactly the same as the later derived multi year form eq.~(\ref{equ:multi_stack}).
%   For these we can simply get all our detector weights for the corresponding year and also need to use the correct PDFs for each source.
%   Then the weights are normalized over all samples.
% \end{enumerate}

% \subsection{How \lstinline|grbllh| does it}
% In \lstinline|grbllh| we do something similar to eq.~(\ref{equ:multi_stack}): \begin{equation}
%   \frac{1}{2}\Lambda
%   = -\hat{n}_S + \sum_{k=1}^{N_\text{srcs}}\sum_{i=1}^{N} \ln\left(
%         \frac{\hat{n}_S S_{i,k}}{\Braket{n_B} B_{i,k}} + 1 \right)
%    \mcomma
% \end{equation}
% which means we insert the total sum of all signal expectations $\hat{n}_S$ and the sum of all background expectations $\Braket{n_B}$.
% This has the same effect as using
% \begin{equation}
%   \frac{1}{2}\Lambda
%   = -\hat{n}_S + \sum_{k=1}^{N_\text{srcs}}\sum_{i=1}^{N} \ln\left(
%         \frac{\hat{n}_S w_k S_{i,k}}{\overline{\Braket{n_B}} B_{i, k}} + 1 \right)
%    \mcomma
% \end{equation}
% where all the signal weights are
% \begin{equation}
%   w_k = 1/N_\text{srcs}
% \end{equation}
% and
% \begin{equation}
%   \overline{\Braket{n_B}} =
%   \frac{1}{N_\text{srcs}}\sum_{k=1}^{N_\text{srcs}}\Braket{n_{B,k}}
%   = \frac{\Braket{n_B}}{N_\text{srcs}}
% \end{equation}
% means the mean background from all source time windows.
% So we simply use a special weighted case, where all GRBs are treated the same with respect to signal and background expectations.


% \subsection{Example Weight Matrix}
% An arbitrary example with 3 samples and 5 sources in total would be:
% \begin{align}
% \vec{w} =
%   \begin{pmatrix}
%     w_{j=0} \\ w_{j=1} \\ w_{j=2}
%   \end{pmatrix} &=
%     \begin{pmatrix}
%       0 & 0 & 1 & 0 & 0 \\
%       0 & 1 & 0 & 0 & 0 \\
%       1 & 0 & 0 & 1 & 1
%     \end{pmatrix} \cdot
%     \begin{pmatrix}
%       w_0^\text{D} w_0^\text{T} \\
%       w_1^\text{D} w_1^\text{T} \\
%       w_2^\text{D} w_2^\text{T} \\
%       w_3^\text{D} w_3^\text{T} \\
%       w_4^\text{D} w_4^\text{T}
%     \end{pmatrix} \cdot
%     \frac{1}{\sum_{k=1}^{N_\text{srcs}} w^\text{D}_k w^\text{T}_k} \\
%     &= \begin{pmatrix}
%         w_2^\text{D} w_2^\text{T} \\
%         w_1^\text{D} w_1^\text{T} \\
%         w_0^\text{D} w_0^\text{T} + w_3^\text{D} w_3^\text{T} +
%         w_4^\text{D} w_4^\text{T}
%       \end{pmatrix} \cdot
%       \frac{1}{\sum_{k=1}^{N_\text{srcs}} w^\text{D}_k w^\text{T}_k}
%     \mperiod
%   \label{equ:multi_example}
% \end{align}

% %%%%%%%%%%%%%%%%%%%%%%%%%%%%%%%%%%%%%%%%%%%%%%%%%%%%%%%%%%%%%%%%%%%%%%%%%%%%%
% %% General method to get a-priori weights from effective areas
% %%%%%%%%%%%%%%%%%%%%%%%%%%%%%%%%%%%%%%%%%%%%%%%%%%%%%%%%%%%%%%%%%%%%%%%%%%%%%
\section{A-priori weight selection}
The a-priori weights for the expected signal of source in a specific sample should match the true emission model as closely as possible.
Note that the weights resemble the expected flux at the detector.
For example, an IceCube like detector that only measures the northern sky, all weights for sources on the southern sky would be zero, regardless of the true intrinsic emission strength because no signal is ever detected.

To construct the weights, that actually distribute the total signal to the expected contribution from each source, signal simulation can be used to estimate the signal efficiencies.
Depending on the assumed per event PDFs, the weights might dependent on any source or any event parameter.
For the spatial term, the weights are dependent only on the source positions and the signal response of the detector at their positions.
The assumption that IceCube is right-ascension symmetric yields source declination dependent stacking weights.
When using time information the weights also depend on the current events time.
Because the expected source emissions can, and usually do vary in time, the signal expectation can change for each event.
Because in the usual construction, each events energy PDF doesn't depend on any source parameter, the weights are also not derived from an energy parameter.

Assuming a spatial and temporal dependence on source parameters the general $n_S$ splitting weight per source $k$ would then be
\begin{equation}
  w_k = w(\vec{x}_k, t, t_k) = w^S(\vec{x}_k)\cdot w^T(t, t_k)
  \mcomma
\end{equation}
where $t$ can be any point in time.
Because we only evaluate the weights for the specific events times, weights are then only evaluated at $t=t_i$.

In the special case treated here, having sources unique in their time windows, the weight calculation simplifies to the spatial, source declination dependent part only, because only one source at a time has an emission expectation, so the normalized time dependent weights are $w_k^T = 1$ for each source.
The general case would be to get the time dependent weights for each event from the time signal PDF per source and then normalizing over all contributions.

For actually calculating the expected number of events $N$ at detector level and for the final sample event selection from an intrinsic neutrino flux $\phi$
\begin{equation}
  N
  = T\cdot\int_0^\infty\int_\Omega
    A_\text{eff}(E,\Omega)\cdot\phi(E,\Omega) \,\d{E}\d{\Omega}
\end{equation}
is used.
$A_\text{eff}$ is the hypothetical detector surface area which would yield the same number of observed events if it would detect $\SI{100}{\percent}$ of the incoming flux compared to the real detector and the applied event selection and maps an intrinsic flux to event counts detected at detector level.
$A_\text{eff}$ is typically in units $\si{\m\squared}$, the flux $\phi$ in \si{\per\GeV\per\m\squared\per\second\per\steradian} and the detector livetime $T$ in $\si{\second}$.
When using a a time integrated fluence $\Phi$ model the livetime is obsolete and the number of events read
\begin{equation}
  N
  = \int_0^\infty\int_\Omega
    A_\text{eff}(E,\Omega)\cdot\Phi(E,\Omega) \,\d{E}\d{\Omega}
  \mperiod
\end{equation}
The expected number of events for a given source declination can then be obtained by integrating over the energy only, getting the number of events per declination.
The normalized weights per source in a single sample can be constructed by
\begin{equation}
  w^S(\vec{x}_k)
  = \frac
      {
        \int_0^\infty A_\text{eff}(E,\Omega_k)\cdot\Phi(E,\Omega_k)\,\d{E}
      }
      {
        \sum_{m=1}^{N_\text{srcs}}
        \int_0^\infty A_\text{eff}(E,\Omega_m)\cdot\Phi(E,\Omega_m)\,\d{E}
      }
  \mperiod
\end{equation}
This effective area formulation holds for calculating the multi sample splitting weights as well as for the signal splitting within a single sample, because it generally calculates the detector efficiency to a given flux or fluence hypothesis.

Additional intrinsic source weights can be introduced to capture differences in expected flux or fluence from the sources themselves.
This is decoupled from the actual detection mechanism and the weights describes above.
The weight selection strongly depends on the used catalogue and can for example be the gamma or x-ray flux or distance weighting.
These intrinsic source weights $w_k^\text{src}$ are independently multiplied with the corresponding detector weights $w_k$ to form the total source weights
\begin{equation}
  w_k^\text{(tot)} \coloneqq w_k \cdot w_k^\text{src}
  \mperiod
\end{equation}


% %%%%%%%%%%%%%%%%%%%%%%%%%%%%%%%%%%%%%%%%%%%%%%%%%%%%%%%%%%%%%%%%%%%%%%%%%%%%%
% %% Derivation of classci time integrated LLH used within skylab
% %%%%%%%%%%%%%%%%%%%%%%%%%%%%%%%%%%%%%%%%%%%%%%%%%%%%%%%%%%%%%%%%%%%%%%%%%%%%%
\section{Time integrated Likelihood}
The standard time integrated Likelihood formula used in IceCube point source searches can also be derived from the general extended Likelihood form.
A different approximation than in the time dependent case is used, which takes into account the usually larger statistics in a time integrated sample as all events count and not only these in temporal coincidence with any source.

\subsection{Single sample stacking case}
Starting again from the general form
\begin{equation}
  -2\ln\Lambda
  = -2\left(\sum_{k=1}^{N_\text{srcs}}\hat{\lambda}_{k,S} +
                                      \hat{\lambda}_{k,B} -
                                      \hat{\lambda}_{k,B}^{(0)}\right) +
    2\sum_{i=1}^N \ln\left(
      \frac{\sum_{k=1}^{N_\text{srcs}}\left(
          \hat{\lambda}_{k,S} S_{i,k}+\hat{\lambda}_{k,B} B_{i,k}\right)}
          {\sum_{k=1}^{N_\text{srcs}}\left(
            \hat{\lambda}_{k,B}^{(0)} B_{i,k}
          \right)}
        \right)
  \mcomma
\end{equation}
the following approximation can be used
\begin{equation}
  \sum_{k=1}^{N_\text{srcs}} \hat{\lambda}_{k,S} + \hat{\lambda}_{k,B} \approx
    \sum_{k=1}^{N_\text{srcs}} \hat{\lambda}_{k,B}^{(0)} \approx N
  \mperiod
\end{equation}
This means the Poisson fluctuations of the sample size are neglected.
Also the second part is usually valid if the amount of signal expected in the data is small compared to the amount of background-like events.

These approximations cancel the Poisson term in front of the sum and leaves
\begin{equation}
  -2\ln\Lambda
  = 2\sum_{i=1}^N \ln\left(
    \frac{\sum_{k=1}^{N_\text{srcs}}\left(
          \hat{\lambda}_{k,S} S_{i,k}+\hat{\lambda}_{k,B} B_{i,k}\right)}
          {\sum_{k=1}^{N_\text{srcs}}\left(
            \hat{\lambda}_{k,B}^{(0)} B_{i,k} \right)}
        \right)
  \mperiod
\end{equation}
For the time independent part, the background distributions $B_{i,k}$ are all the same, because they only depend on each event's location and not on any source related parameters any more.
Thus the estimators and PDFs can be written as
\begin{equation}
  \hat{\lambda}_{k,B} = \frac{1}{N_\text{srcs}} \hat{\lambda}_B
  \mintertext{and} B_{i,k} = B_i
\end{equation}
and the denominator in the logarithm can be simplified to
\begin{equation}
  \frac{1}{\sum_{k=1}^{N_\text{srcs}}\left(
           \hat{\lambda}_{k,B}^{(0)}B_{i,k} \right)}
  = \frac{1}{B_i \sum_{k=1}^{N_\text{srcs}}\hat{\lambda}_{k,B}^{(0)}}
  = \frac{1}{N B_i}
  \mperiod
\end{equation}
Using the same argument for the background distributions in the nominator, the stacking test statistic gets
\begin{align}
  -2\ln\Lambda
  &= 2\sum_{i=1}^N \ln\left(
    \frac{\sum_{k=1}^{N_\text{srcs}}\left(
          \hat{\lambda}_{k,S} S_{i,k}+\hat{\lambda}_{k,B} B_{i,k}\right)}
          {N B_i} \right) \\
  &= 2\sum_{i=1}^N \ln\left(
    \frac{\sum_{k=1}^{N_\text{srcs}}\left(\hat{\lambda}_{k,S} S_{i,k}\right) +
          \hat{\lambda}_B B_i}{N B_i} \right)
  \mperiod
\end{align}

Using the fixed expectation approximation from above again, the background parameter $\hat{\lambda}_B$ can be eliminated leaving $N_\text{srcs}$ free signal parameters, one for each source
\begin{align}
  -2\ln\Lambda
  &= 2\sum_{i=1}^N \ln\left(
    \frac{\sum_{k=1}^{N_\text{srcs}}\left(\hat{\lambda}_{k,S} S_{i,k}\right) +
          \left(
            N - \sum_{k=1}^{N_\text{srcs}}\hat{\lambda}_{k,S}
          \right) B_i}{N B_i} \right) \\
  &= 2\sum_{i=1}^N \ln\left(
      \frac{\sum_{k=1}^{N_\text{srcs}}\left(\hat{\lambda}_{k,S}
              S_{i,k}\right)}{N B_i} -
      \frac{\sum_{k=1}^{N_\text{srcs}}\hat{\lambda}_{k,S}}{N} + 1
    \right)
  \mperiod
\end{align}
where in the last step the equation is slightly rearranged to show the commonly used test statistic formula in the general time integrated stacking case.

The $N_\text{srcs}$ free parameters $\lambda_{k,S}$ can again be reduced to a single signal strength parameter $n_S$ when a-priori knowledge about the source class proportions is used via
\begin{equation}
  \lambda_{k,S} = n_S \cdot w_k
  \mintertext{and}
  \sum_{k=1}^{N_\text{srcs}} w_k = 1
  \mperiod
\end{equation}
The test statistic can then be further reduced to
\begin{align}
  -2\ln\Lambda
  &= 2\sum_{i=1}^N \ln\left(
        \frac{\hat{n}_S \sum_{k=1}^{N_\text{srcs}}\left(w_k S_{i,k}\right)}
             {N B_i} -
        \frac{\hat{n}_S \sum_{k=1}^{N_\text{srcs}}w_k}{N} + 1
      \right) \\
  &= 2\sum_{i=1}^N \ln\left(
        \frac{\hat{n}_S}{N}\left(
          \frac{\sum_{k=1}^{N_\text{srcs}}(w_k S_{i,k})}{B_i} - 1
        \right) + 1
      \right)
  \mperiod
\end{align}
Sometimes the signal sum term is abbreviated to
\begin{equation}
  S_i^\text{(tot)} \coloneqq \sum_{k=1}^{N_\text{srcs}}\left(w_k S_{i,k}\right)
\end{equation}
and only in this case with a-priori fixed weights $w_k$, the often stated \enquote{simple replacement of the single source signal term $S_i$ with the summed signal term $S_i^\text{(tot)}$} is valid
\begin{equation}
  \sum_{i=1}^N \ln\left(
      \frac{\hat{n}_S}{N}\left( \frac{S_i}{B_i} - 1 \right) + 1
    \right)
  \rightarrow
  \sum_{i=1}^N \ln\left(
      \frac{\hat{n}_S}{N}\left( \frac{S_i^\text{(tot)}}{B_i} - 1 \right) + 1
    \right)
  \mperiod
\end{equation}

\subsection{Multiple samples stacking case}
Construction of the multi sample Likelihood formula in the time integrated case is done exactly as in the time dependent case before by starting with the sample weight relation
\begin{equation}
  w_j = P(j) = \sum_{k=1}^{N_\text{srcs}} P(j|k)\cdot P(k)
  \mperiod
\end{equation}
Because all sources contribute in all times and thus all samples, the weights are not written out in a compact form here, but they still represent the relative sensitivity of a single sample to a given source hypothesis combined with the global sensitivity to a single source across all samples.
This may also be seen as a special case for the time dependent Likelihood weights, where all the time windows for signal and background are as large as the sample livetimes.

A notable difference arises, if another global fit parameter is introduced alongside the mandatory $n_S$.
This is often done to better adapt to the unknown signal hypothesis and in a time integrated analysis the statistics are usually sufficient to reliably fit an additional parameter.
Usually this a parameter describing the signal flux hypothesis and is modelled as a single unbroken power law with spectral index $\gamma$ in the energy PDF term
\begin{equation}
  w^S(\vec{x}_k, \gamma)
  = \frac
      {
        \int_0^\infty A_\text{eff}(E,\Omega_k)\cdot
          \Phi(E, \Omega_k, \gamma)\,\d{E}
      }
      {
        \sum_{m=1}^{N_\text{srcs}}
        \int_0^\infty A_\text{eff}(E,\Omega_m)\cdot
          \Phi(E, \Omega_m, \gamma)\,\d{E}
      }
  \mperiod
\end{equation}
This leads to splitting weights that dependent on the actual shape of the assumed signal flux, because the sensitivity per sample can change when the flux gets harder or softer.

% \listofnotes
