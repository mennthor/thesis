\chapter{Introduction}

Since the first proposal in  1960 to use underwater apparatus to detect Cerenkov light from charged particles originating from astrophysical neutrinos, neutrino astronomy has come a long way \CITE{markov}.
In 2013, three years after operating in its final configuration, the IceCube neutrino observatory found the first evidence of a diffuse, astrophysical neutrino flux \CITE{science, IC detector paper}.
Giving the possibility to probe a broad spectrum of astrophysical and particle physics topics, one of the primary goals of the IceCube detector has been the direct identification of the sources of astrophysical neutrinos ever since \CITE{Any physics overview paper}.
Despite numerous efforts, including general searches of the whole sky, stacked approaches testing various source catalogues and tests for emission from a whole region like the galactic plane with different event topologies, no significant source of neutrinos have been found so far \CITE{A whole bunch of PS papers here}.
This changed with the very recent discovery of a flaring Blazar coincident with an ultra-high-energy neutrino, measured on September 22nd, 2017, with the IceCube neutrino observatory and the Fermi satellite \CITE{TXS paper}.
Despite the correlation between these two events, a scan of earlier neutrino data revealed a significant evidence of a neutrino flux coming from the Blazar's position emitted on a 100-day long timescale \CITE{TXS PS paper}.

However fortunate this detection was, a single few neutrinos and a single assigned source cannot unveil all of the unknown properties of neutrino production mechanisms and the underlying cosmic ray acceleration processes.
This justifies the ongoing efforts to further identify and investigate the astrophysical neutrino emission scenarios and discover the fundamental physics at play.
Additionally, it shows that a more comprehensive picture of the internal workings of astrophysical sources can be achieved by the combination of multiple observation channels.
Having observatories covering a broad range of multiple messenger particles, including photons in almost every wavelength \CITE{photon astro review}, charged cosmic rays \CITE{any CR review}, the aforementioned neutrinos and most recently even gravitational waves, which are already incorporated in correlation searches \CITE{LIGO IC and perhaps LIGO other paper}.

In this thesis, a similar approach is followed.
Similar to photons and cosmic rays, also neutrinos are identified in a broad range and,  after now almost eight years of full capacity data taking, in great abundance by the IceCube detector.
Instead of testing for correlation with other messenger particles, it is, therefore, possible to try to pinpoint a possible correlation between high energy starting events and a larger set of lower energy neutrinos.
The latter neutrinos are measured with greater abundance, but it cannot be safely concluded, if a single event is of astrophysical origin or not, due to large amounts of background events for lower energies.
For the former high energy events, it can be quite safely concluded that they are of astrophysical origin, for most of them even on a per-event scale.
Therefore, if a correlation between lower energy events can be identified, it strengthens the ground for further multi-messenger observation at these specific source locations, to identify the underlying sources with great precision.

In this thesis, two emission scenarios for a combined, lower energy neutrino clustering at the positions of 22 high energy starting events, measured in six years of IceCube data is presented.
It is tested for both a time-dependent emission scenario on short to medium time ranges and a time-independent, steady-state flux scenario.
The second chapter gives a short overview of neutrino astroparticle physics and possible emission models.
In the third chapter, the IceCube detector is described and the detection method explained.
The fourth chapter briefly describes the used datasets, consisting of the source data set of high energy starting events and the test data set, made of six years of all-sky muon neutrino track events, that make up the primary channel for point source searches.

These chapters are held quite short, because of the great abundance of material dealing in great detail with many aspects of these topics.
In this thesis, the focus lies more on the methodical approaches and detailed description of the analysis structure.
Therefore, in chapter 5, a detailed derivation of the Likelihood formulas for both analyses is given.
In chapters six and seven, both analyses for the time-dependent and the steady state emission scenario are described.
These chapters explain in detail how choices for constructing probability density functions or weights used in the Likelihood formulations are made and should hopefully enable to obtain a good understanding of how such analysis work.
Chapter eight summarizes the results of both analyses and sorts them into the most recent development of searches for astrophysical neutrino sources.
The conclusion and outlook can be found in chapter nine and closes this thesis.
