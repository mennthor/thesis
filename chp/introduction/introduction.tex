\chapter{Introduction}

Since the first proposal in  1960 to use underwater apparatus to detect Cherenkov light from charged particles originating from astrophysical neutrinos, neutrino astronomy has come a long way \cite{Markov:1960vja}.
In 2013, three years after the IceCube neutrino observatory was operating in its final configuration, the first evidence of a diffuse, astrophysical neutrino flux was found in IceCube data \cite{Aartsen:2013jdh,Aartsen:2016nxy}.
Although giving the opportunity to probe a broad spectrum of astrophysical and particle physics topics, one of the primary goals of the IceCube detector has been the direct identification of the sources of astrophysical neutrinos ever since \cite{Katz:2011ke,GOLDSCHMIDT:2002ICScienceGoals}.
Despite numerous efforts, including general searches of the whole sky, stacked approaches testing various source catalogues and tests for emission from a extended region like the galactic plane with different event topologies, no significant source of neutrinos have been found so far \cite{Ahlers:2014ioa,Aartsen:2016oji,Aartsen:2013jdh,Aartsen:2016qcr,Adrian-Martinez:2015ver,Aartsen:2016tpb,IceCube:2018cha,Aartsen:2015dml,Abbasi:2009kq,Aartsen:2014PS4yrs,Aartsen:2015wto}.
This changed with the very recent discovery of a Blazar flare, measured with the Fermi satellite, coincident with the arrival direction of an extremely high energy neutrino, measured with the IceCube neutrino observatory on September 22nd, 2017 \cite{Keivani:2018rnh}.
In addition to the correlation between these two events, a scan of earlier neutrino data also revealed evidence of a temporally constrained neutrino flux coming from the Blazar's position emitted on a 100-day long timescale \cite{IceCube:2018cha}.

Despite this detection, the few detected neutrinos coincident with a single source cannot unveil all of the unknown properties of neutrino production mechanisms and the underlying cosmic ray acceleration processes.
This justifies the ongoing efforts to further investigate the astrophysical neutrino emission scenarios and identify the fundamental physics at play during the neutrino creation in the source regions.
Additionally, it shows that a more comprehensive picture of the internal workings of astrophysical sources can be achieved by the combination of multiple observation channels.
Having a collection of observatories that together are able to detect a broad range of astrophysical messengers, including photons in almost every wavelength \cite{Huber:2016Photons}, charged cosmic rays \cite{BLANDFORD:2014CRs,Sommers:2009CRs}, the aforementioned neutrinos and most recently even gravitational waves, which are already incorporated in correlation searches \cite{Abbott:2016blz,Adrian-Martinez:2016xgn}, the way is paved for powerful new analyses \cite{Branchesi:2016MultiMessenger}.

In this thesis, a similar approach is followed.
Equivalently to photons and cosmic rays, neutrinos are also measured in a broad energy range and, after having data available from eight years of operation in its full configuration state, also in great abundance, by the IceCube detector.
Instead of testing for correlation with other messenger particles, it is also feasible to try to pinpoint a possible correlation between high energy starting events and a larger set of lower energy neutrino events.
The lower energy neutrinos are measured with great abundance but it cannot be safely concluded if a single event is of astrophysical origin or not, due to the large amounts of background events for lower energies.
For the aforementioned high energy starting events on the other hand, it can be quite safely concluded that they are of astrophysical origin, for most of them even on a per-event scale.
Therefore, if a correlation between lower energy events can be identified, it strengthens the case for further efforts in multi-messenger observations at these specific source locations, to identify the underlying sources with better precision.

In this thesis, two analyses for different emission scenarios for a combined, lower energy neutrino clustering at the positions of 22 track-like high energy starting events, measured in six years of IceCube data are presented.
It is tested for both a time-dependent emission scenario on short to medium time ranges and a time-independent, steady-state flux scenario.
The second chapter gives a short overview of neutrino astroparticle physics and some prominent astrophysical source emission models.
In the third chapter, the IceCube detector is described and the detection method explained.
The fourth chapter briefly describes the used datasets, consisting of the source data set of track-like high energy starting events and the test data set, made of six years of all-sky muon neutrino track events, that represent the primary channel for neutrino point source searches.

These chapters are held quite short, because of the great abundance of material available dealing in great detail with many aspects of these topics.
In this thesis, the focus lies more on the methodical approaches and detailed description of the analysis structure.
Therefore, in chapter 5, a detailed derivation of the Likelihood formulas for both analyses is given.
In chapters six and seven, both analyses for the time-dependent and the steady state emission scenario are described.
These chapters explain in detail how choices for constructing probability density functions or weights used in the Likelihood formulations are made.
The aim is to give a good understanding of how such analyses work and enable the reader to reproduce the main aspects of the presented analyses.
Chapter eight summarizes the results and presents the main physics results of both analyses.
The conclusion and outlook, which sorts them into the most recent development of searches for astrophysical neutrino sources can be found in chapter nine and closes this thesis.
