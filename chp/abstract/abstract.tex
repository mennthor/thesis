\thispagestyle{plain}

\section*{Abstract}
In this thesis, an approach is presented to probe the locations of track-like high energy starting events (HESE) for an additional clustering of neutrino events within six years of IceCube neutrino data at these positions.
Two unbinned Likelihood stacking analyses using both a time-dependent, per-burst and a steady-state neutrino emission scenario are performed.
This is done to be sensitive to a collection of weak fluxes, undetectable on their own and to test a rather broad regime of emission scenarios in the light of the unknown, underlying sources types.
In both analyses, no significant excess of an additional neutrino contribution was found.
However, a slight over-fluctuation of about $\SIsigma{1}$ in the largest tested time window, with a width of five days, and about $\SIsigma{0.7}$ in the time-integrated analysis is measured.
This mildly indicates a connection to the recent measurement of a correlation between the flaring Blazar TXS~0506+056 and an extremely high energy neutrino starting event, where evidence was found for an additional neutrino emission on timescales of about $\num{110}$ days, right in between the tested emission scenarios in this thesis.

\section*{Kurzfassung}
\begin{german}
In dieser Arbeit wird die Suche nach einem Neutrinofluss an den Positionen von hochenergetischen, im Detektor startenden Neutrinoereignissen, \emph{High Energy Starting Events} (HESE), in sechs Jahren IceCube Neutrino Daten beschrieben.
Es wird ein ungebinnter, gestackter Likelihood-Ansatz benutzt, um nach zusätzlichen Neutrinoereignissen zu suchen, die räumlich korreliert zu den HESE-Ereignissen erzeugt wurden.
Da die potentiellen Quellen, aus denen die HESE Ereignisse stammen können unbekannt sind, wird sowohl ein zeitabhängiges, als auch ein stetiges Emissionsszenario getestet.
In beiden Analysen konnte kein signifikantes Signal gefunden werden.
Allerdings zeigt sowohl die zeitabhängige Messung mit etwa $\SIsigma{1}$ im größten getesteten Zeitfenster, mit einer Breite von fünf Tagen, als auch die zeitintegrierte Messung mit $\SIsigma{0.7}$ eine leichte Überfluktuation.
Beide Ergebnisse lassen sich damit in den Rahmen der kürzlich entdeckten Korrelation eines extrem hochenergetischen Neutrinos und der erhöhten Aktivität des Blazars TXS~0506+056, bei der eine zusätzliche Neutrinoemission auf einer Zeitskala von etwa $\num{110}$ Tagen nachgewiesen wurde, einfügen.
\end{german}