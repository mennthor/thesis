\thispagestyle{plain}

\section*{Abstract}
In the light of the recently discovered connection between a flaring Blazar and an ultra-high energy neutrino event measured with IceCube, an approach is presented to probe the locations of high energy starting events (HESE) for an additional, lower energy clustering of neutrino events at these positions.
In six years, 22 HESE track-like events with a reasonably small angular uncertainty of about one degree were measured by the IceCube detector.
However, no known or new sources in either photon or cosmic ray observations were discovered at follow-up observations or in archival data at these positions.
An additional neutrino signal at these positions would justify the extended search for dim emitters in other wavelengths to further explore neutrino production scenarios and strengthen the astrophysical origin of the high energy starting neutrinos.

An unbinned Likelihood stacking approach using both a time-dependent per-burst and a steady-state neutrino emission scenario is performed in this work to be potentially sensitive to a collection of low fluxes, undetectable on their own and to test a rather broad regime in light of the unknown, underlying sources types.
In both analyses, no significant excess of an additional neutrino contribution could be found.
However, a slight over fluctuation of about $\SI{1}{\sigma}$ is found, mildly indicating a connection to the recent Blazar correlation result, which found evidence for a neutrino emission on timescales of about one hundred days, right in between the tested emission scenarios in this thesis.

\section*{Kurzfassung}
\begin{german}
In dieser Arbeit wird die Suche nach einem Neutrinofluss an den Positionen von hoch energetischen, im Detektor startenden Neutrinoereignissen (\enquote{High Energy Starting Events}, HESE) beschrieben.
Im Kontext der kürzlichen Entdeckung eines Zusammenhangs zwischen einem stark emittierenden Blazar und einem hochenergetischen, startend Neutrinoereigniss in IceCube ist die Suche nach Korrelationen zwischen verschiedensten, astrophysikalischen Botenteilchen erneut in den Fokus gerückt.

In sechs Jahren Daten des IceCube Detektors konnten 22 hochenergetische, im Detektor startende Neutrinos gemessen werden.
Allerdings wurden weder in Nachfolgeobservationen anderer Instrumente, noch in Archivdaten bekannte Quellen in anderen Botenteilchen wie Photonen oder geladener, kosmischer Strahlung gefunden.
Ein zusätzliches Neutrino Signal an diesen Ereignisspositionen würde die weiterführende Suche an anderen Observatorien motivieren und einen weiteren Schritt in Richtung des Verständnisses der Erzeugungsmechanismen von Neutrinos in astrophysikalischen Quellen beitragen.

Es wird ein ungebinnter, gestackter Likelihood Ansatz benutzt, um nach zusätzlichen Neutrinoereignissen zu suchen, die räumlich korreliert zu den HESE Ereignissen erzeugt wurden.
Da die potentiellen Quellen, aus denen die HESE Ereignisse stammen können unbekannt sind, wird sowohl ein zeitabhängiges als auch ein stetiges Emissionsszenario getestet.
In beiden Analysen konnte kein signifikantes Signal gefunden werden.
Allerdings zeigen beide Analysen eine leichte Überfluktuation von etwa $\SI{1}{\sigma}$, welche sich in den Rahmen der kürzlich entdeckten Blazar Korrelation fügt, die eine signifikante Neutrinoemission auf einer Zeitskala von etwa 100 Tagen nachgewiesen hat.
\end{german}